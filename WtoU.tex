\documentclass[a4paper, reqno, 12pt]{amsart}
% alternative choice:
%\documentclass[a4paper, reqno, 12pt]{article}
%\usepackage{authblk}    %comment this out when using amsart


\usepackage{amsthm,amsfonts,amssymb,amsmath,amsxtra,amsrefs,array,dynkin-diagrams}
\usepackage{xr-hyper}
\usepackage[matrix,tips,frame,color,line,poly]{xy}
%\usepackage[colorlinks=
%   citecolor=Black,
%   linkcolor=Red,
%   urlcolor=Blue]{hyperref}
\usepackage{verbatim}
\usepackage{tikz}
\usepackage[margin=1.25in]{geometry}
\usepackage{mathrsfs}
%\usepackage{showkeys}
\setlength\extrarowheight{6pt}  %extra height in tables, using array package
\newcommand{\remind}[1]{{\bf ** #1 **}}

\RequirePackage{xspace}
% load etoolbox package, for programming features
\RequirePackage{etoolbox}
% load varwidth package, for text environments which are automatically the natural width of the text they contain
\RequirePackage{varwidth}
% load enumitem package, for easy margin adjustment in enumerate and itemize environments
\RequirePackage{enumitem}
% load tensor package, for good placement of super/subscripts to the left of symbols
\RequirePackage{tensor}
% load mathtools package, for various extensions of amsmath
\RequirePackage{mathtools}
% load longtable package, which allows tables to (if needed) split over multiple pages
\RequirePackage{longtable}
% load multirow package, which allows cells spanning multiple rows in tables
\RequirePackage{multirow}



% put sections only (as opposed to subsections) in the table of contents
\setcounter{tocdepth}{1}


\def\ge{\geqslant}
\def\le{\leqslant}
\def\a{\alpha}
\def\b{\beta}
\def\g{\gamma}
\def\G{\Gamma}
\def\d{\delta}
\def\D{\Delta}
\def\L{\Lambda}
\def\e{\epsilon}
\def\et{\eta}
\def\io{\iota}
\def\o{\omega}
\def\p{\pi}
\def\ph{\phi}
\def\ps{\psi}
%\def\r{\rho}
\def\s{\sigma}
\def\t{\tau}
\def\th{\theta}
\def\k{\kappa}
\def\l{\lambda}
\def\z{\zeta}
\def\v{\vartheta}
\def\x{\xi}
\def\i{^{-1}}

\def\<{\langle}
\def\>{\rangle}

\newcommand{\sA}{\ensuremath{\mathscr{A}}\xspace}
\newcommand{\sB}{\ensuremath{\mathscr{B}}\xspace}
\newcommand{\sC}{\ensuremath{\mathscr{C}}\xspace}
\newcommand{\sD}{\ensuremath{\mathscr{D}}\xspace}
\newcommand{\sE}{\ensuremath{\mathscr{E}}\xspace}
\newcommand{\sF}{\ensuremath{\mathscr{F}}\xspace}
\newcommand{\sG}{\ensuremath{\mathscr{G}}\xspace}
\newcommand{\sH}{\ensuremath{\mathscr{H}}\xspace}
\newcommand{\sI}{\ensuremath{\mathscr{I}}\xspace}
\newcommand{\sJ}{\ensuremath{\mathscr{J}}\xspace}
\newcommand{\sK}{\ensuremath{\mathscr{K}}\xspace}
\newcommand{\sL}{\ensuremath{\mathscr{L}}\xspace}
\newcommand{\sM}{\ensuremath{\mathscr{M}}\xspace}
\newcommand{\sN}{\ensuremath{\mathscr{N}}\xspace}
\newcommand{\sO}{\ensuremath{\mathscr{O}}\xspace}
\newcommand{\sP}{\ensuremath{\mathscr{P}}\xspace}
\newcommand{\sQ}{\ensuremath{\mathscr{Q}}\xspace}
\newcommand{\sR}{\ensuremath{\mathscr{R}}\xspace}
\newcommand{\sS}{\ensuremath{\mathscr{S}}\xspace}
\newcommand{\sT}{\ensuremath{\mathscr{T}}\xspace}
\newcommand{\sU}{\ensuremath{\mathscr{U}}\xspace}
\newcommand{\sV}{\ensuremath{\mathscr{V}}\xspace}
\newcommand{\sW}{\ensuremath{\mathscr{W}}\xspace}
\newcommand{\sX}{\ensuremath{\mathscr{X}}\xspace}
\newcommand{\sY}{\ensuremath{\mathscr{Y}}\xspace}
\newcommand{\sZ}{\ensuremath{\mathscr{Z}}\xspace}


\newcommand{\fka}{\ensuremath{\mathfrak{a}}\xspace}
\newcommand{\fkb}{\ensuremath{\mathfrak{b}}\xspace}
\newcommand{\fkc}{\ensuremath{\mathfrak{c}}\xspace}
\newcommand{\fkd}{\ensuremath{\mathfrak{d}}\xspace}
\newcommand{\fke}{\ensuremath{\mathfrak{e}}\xspace}
\newcommand{\fkf}{\ensuremath{\mathfrak{f}}\xspace}
\newcommand{\fkg}{\ensuremath{\mathfrak{g}}\xspace}
\newcommand{\fkh}{\ensuremath{\mathfrak{h}}\xspace}
\newcommand{\fki}{\ensuremath{\mathfrak{i}}\xspace}
\newcommand{\fkj}{\ensuremath{\mathfrak{j}}\xspace}
\newcommand{\fkk}{\ensuremath{\mathfrak{k}}\xspace}
\newcommand{\fkl}{\ensuremath{\mathfrak{l}}\xspace}
\newcommand{\fkm}{\ensuremath{\mathfrak{m}}\xspace}
\newcommand{\fkn}{\ensuremath{\mathfrak{n}}\xspace}
\newcommand{\fko}{\ensuremath{\mathfrak{o}}\xspace}
\newcommand{\fkp}{\ensuremath{\mathfrak{p}}\xspace}
\newcommand{\fkq}{\ensuremath{\mathfrak{q}}\xspace}
\newcommand{\fkr}{\ensuremath{\mathfrak{r}}\xspace}
\newcommand{\fks}{\ensuremath{\mathfrak{s}}\xspace}
\newcommand{\fkt}{\ensuremath{\mathfrak{t}}\xspace}
\newcommand{\fku}{\ensuremath{\mathfrak{u}}\xspace}
\newcommand{\fkv}{\ensuremath{\mathfrak{v}}\xspace}
\newcommand{\fkw}{\ensuremath{\mathfrak{w}}\xspace}
\newcommand{\fkx}{\ensuremath{\mathfrak{x}}\xspace}
\newcommand{\fky}{\ensuremath{\mathfrak{y}}\xspace}
\newcommand{\fkz}{\ensuremath{\mathfrak{z}}\xspace}


\newcommand{\fkA}{\ensuremath{\mathfrak{A}}\xspace}
\newcommand{\fkB}{\ensuremath{\mathfrak{B}}\xspace}
\newcommand{\fkC}{\ensuremath{\mathfrak{C}}\xspace}
\newcommand{\fkD}{\ensuremath{\mathfrak{D}}\xspace}
\newcommand{\fkE}{\ensuremath{\mathfrak{E}}\xspace}
\newcommand{\fkF}{\ensuremath{\mathfrak{F}}\xspace}
\newcommand{\fkG}{\ensuremath{\mathfrak{G}}\xspace}
\newcommand{\fkH}{\ensuremath{\mathfrak{H}}\xspace}
\newcommand{\fkI}{\ensuremath{\mathfrak{I}}\xspace}
\newcommand{\fkJ}{\ensuremath{\mathfrak{J}}\xspace}
\newcommand{\fkK}{\ensuremath{\mathfrak{K}}\xspace}
\newcommand{\fkL}{\ensuremath{\mathfrak{L}}\xspace}
\newcommand{\fkM}{\ensuremath{\mathfrak{M}}\xspace}
\newcommand{\fkN}{\ensuremath{\mathfrak{N}}\xspace}
\newcommand{\fkO}{\ensuremath{\mathfrak{O}}\xspace}
\newcommand{\fkP}{\ensuremath{\mathfrak{P}}\xspace}
\newcommand{\fkQ}{\ensuremath{\mathfrak{Q}}\xspace}
\newcommand{\fkR}{\ensuremath{\mathfrak{R}}\xspace}
\newcommand{\fkS}{\ensuremath{\mathfrak{S}}\xspace}
\newcommand{\fkT}{\ensuremath{\mathfrak{T}}\xspace}
\newcommand{\fkU}{\ensuremath{\mathfrak{U}}\xspace}
\newcommand{\fkV}{\ensuremath{\mathfrak{V}}\xspace}
\newcommand{\fkW}{\ensuremath{\mathfrak{W}}\xspace}
\newcommand{\fkX}{\ensuremath{\mathfrak{X}}\xspace}
\newcommand{\fkY}{\ensuremath{\mathfrak{Y}}\xspace}
\newcommand{\fkZ}{\ensuremath{\mathfrak{Z}}\xspace}




\newcommand{\heart}{{\heartsuit}}
\newcommand{\club}{{\clubsuit}}
\newcommand{\diam}{{\Diamond}}
\newcommand{\spade}{{\spadesuit}}

\newcommand{\bA}{\mathbf A}
\newcommand{\bE}{\mathbf E}
\newcommand{\bG}{\mathbf G}
\newcommand{\bK}{\mathbf K}
\newcommand{\bM}{\mathbf M}
\newcommand{\bQ}{\mathbf Q}



\newcommand{\BA}{\ensuremath{\mathbb {A}}\xspace}
\newcommand{\BB}{\ensuremath{\mathbb {B}}\xspace}
\newcommand{\BC}{\ensuremath{\mathbb {C}}\xspace}
\newcommand{\C}{\BC} % use this a lot
\newcommand{\BD}{\ensuremath{\mathbb {D}}\xspace}
\newcommand{\BE}{\ensuremath{\mathbb {E}}\xspace}
\newcommand{\BF}{\ensuremath{\mathbb {F}}\xspace}
\newcommand{{\BG}}{\ensuremath{\mathbb {G}}\xspace}
\newcommand{\BH}{\ensuremath{\mathbb {H}}\xspace}
\newcommand{\BI}{\ensuremath{\mathbb {I}}\xspace}
\newcommand{\BJ}{\ensuremath{\mathbb {J}}\xspace}
\newcommand{{\BK}}{\ensuremath{\mathbb {K}}\xspace}
\newcommand{\BL}{\ensuremath{\mathbb {L}}\xspace}
\newcommand{\BM}{\ensuremath{\mathbb {M}}\xspace}
\newcommand{\BN}{\ensuremath{\mathbb {N}}\xspace}
\newcommand{\BO}{\ensuremath{\mathbb {O}}\xspace}
\newcommand{\BP}{\ensuremath{\mathbb {P}}\xspace}
\newcommand{\BQ}{\ensuremath{\mathbb {Q}}\xspace}
\newcommand{\BR}{\ensuremath{\mathbb {R}}\xspace}
\newcommand{\BS}{\ensuremath{\mathbb {S}}\xspace}
\newcommand{\BT}{\ensuremath{\mathbb {T}}\xspace}
\newcommand{\BU}{\ensuremath{\mathbb {U}}\xspace}
\newcommand{\BV}{\ensuremath{\mathbb {V}}\xspace}
\newcommand{\BW}{\ensuremath{\mathbb {W}}\xspace}
\newcommand{\BX}{\ensuremath{\mathbb {X}}\xspace}
\newcommand{\BY}{\ensuremath{\mathbb {Y}}\xspace}
\newcommand{\BZ}{\ensuremath{\mathbb {Z}}\xspace}



\newcommand{\CA}{\ensuremath{\mathcal {A}}\xspace}
\newcommand{\CB}{\ensuremath{\mathcal {B}}\xspace}
\newcommand{\CC}{\ensuremath{\mathcal {C}}\xspace}
\newcommand{\CD}{\ensuremath{\mathcal {D}}\xspace}
\newcommand{\CE}{\ensuremath{\mathcal {E}}\xspace}
\newcommand{\CF}{\ensuremath{\mathcal {F}}\xspace}
\newcommand{\CG}{\ensuremath{\mathcal {G}}\xspace}
\newcommand{\CH}{\ensuremath{\mathcal {H}}\xspace}
\newcommand{\CI}{\ensuremath{\mathcal {I}}\xspace}
\newcommand{\CJ}{\ensuremath{\mathcal {J}}\xspace}
\newcommand{\CK}{\ensuremath{\mathcal {K}}\xspace}
\newcommand{\CL}{\ensuremath{\mathcal {L}}\xspace}
\newcommand{\CM}{\ensuremath{\mathcal {M}}\xspace}
\newcommand{\CN}{\ensuremath{\mathcal {N}}\xspace}
\newcommand{\CO}{\ensuremath{\mathcal {O}}\xspace}
\newcommand{\CP}{\ensuremath{\mathcal {P}}\xspace}
\newcommand{\CQ}{\ensuremath{\mathcal {Q}}\xspace}
\newcommand{\CR}{\ensuremath{\mathcal {R}}\xspace}
\newcommand{\CS}{\ensuremath{\mathcal {S}}\xspace}
\newcommand{\CT}{\ensuremath{\mathcal {T}}\xspace}
\newcommand{\CU}{\ensuremath{\mathcal {U}}\xspace}
\newcommand{\CV}{\ensuremath{\mathcal {V}}\xspace}
\newcommand{\CW}{\ensuremath{\mathcal {W}}\xspace}
\newcommand{\CX}{\ensuremath{\mathcal {X}}\xspace}
\newcommand{\CY}{\ensuremath{\mathcal {Y}}\xspace}
\newcommand{\CZ}{\ensuremath{\mathcal {Z}}\xspace}


\newcommand{\RA}{\ensuremath{\mathrm {A}}\xspace}
\newcommand{\RB}{\ensuremath{\mathrm {B}}\xspace}
\newcommand{\RC}{\ensuremath{\mathrm {C}}\xspace}
\newcommand{\RD}{\ensuremath{\mathrm {D}}\xspace}
\newcommand{\RE}{\ensuremath{\mathrm {E}}\xspace}
\newcommand{\RF}{\ensuremath{\mathrm {F}}\xspace}
\newcommand{\RG}{\ensuremath{\mathrm {G}}\xspace}
\newcommand{\RH}{\ensuremath{\mathrm {H}}\xspace}
\newcommand{\RI}{\ensuremath{\mathrm {I}}\xspace}
\newcommand{\RJ}{\ensuremath{\mathrm {J}}\xspace}
\newcommand{\RK}{\ensuremath{\mathrm {K}}\xspace}
\newcommand{\RL}{\ensuremath{\mathrm {L}}\xspace}
\newcommand{\RM}{\ensuremath{\mathrm {M}}\xspace}
\newcommand{\RN}{\ensuremath{\mathrm {N}}\xspace}
\newcommand{\RO}{\ensuremath{\mathrm {O}}\xspace}
\newcommand{\RP}{\ensuremath{\mathrm {P}}\xspace}
\newcommand{\RQ}{\ensuremath{\mathrm {Q}}\xspace}
\newcommand{\RR}{\ensuremath{\mathrm {R}}\xspace}
\newcommand{\RS}{\ensuremath{\mathrm {S}}\xspace}
\newcommand{\RT}{\ensuremath{\mathrm {T}}\xspace}
\newcommand{\RU}{\ensuremath{\mathrm {U}}\xspace}
\newcommand{\RV}{\ensuremath{\mathrm {V}}\xspace}
\newcommand{\RW}{\ensuremath{\mathrm {W}}\xspace}
\newcommand{\RX}{\ensuremath{\mathrm {X}}\xspace}
\newcommand{\RY}{\ensuremath{\mathrm {Y}}\xspace}
\newcommand{\RZ}{\ensuremath{\mathrm {Z}}\xspace}



\newcommand{\ab}{{\mathrm{ab}}}
\newcommand{\Ad}{{\mathrm{Ad}}}
\newcommand{\ad}{{\mathrm{ad}}}
\newcommand{\alb}{{\mathrm{alb}}}
\DeclareMathOperator{\Aut}{Aut}

\newcommand{\Br}{{\mathrm{Br}}}

\newcommand{\cay}{\ensuremath{\operatorname{\fkc_\xi}}\xspace}
\newcommand{\Ch}{{\mathrm{Ch}}}
\DeclareMathOperator{\charac}{char}
\DeclareMathOperator{\Coker}{Coker}
\newcommand{\cod}{{\mathrm{cod}}}
\newcommand{\cont}{{\mathrm{cont}}}
\newcommand{\cl}{{\mathrm{cl}}}
\newcommand{\Cl}{{\mathrm{Cl}}}
\newcommand{\cm}{{\mathrm {cm}}}
\newcommand{\corr}{\mathrm{corr}}

\newcommand{\del}{\operatorname{\partial Orb}}
\DeclareMathOperator{\diag}{diag}
\newcommand{\disc}{{\mathrm{disc}}}
\DeclareMathOperator{\dist}{dist}
\newcommand{\Div}{{\mathrm{Div}}}
\renewcommand{\div}{{\mathrm{div}}}
\newcommand{\DR}{\mathrm{DR}}

\DeclareMathOperator{\End}{End}

\newcommand{\Fil}{\ensuremath{\mathrm{Fil}}\xspace}
\DeclareMathOperator{\Frob}{Frob}

\DeclareMathOperator{\Adm}{Adm}
\DeclareMathOperator{\EO}{EO}
\DeclareMathOperator{\EOfin}{EO_{\rm fin}}






\DeclareMathOperator{\Gal}{Gal}
\newcommand{\Ztwo}{\BZ/2\BZ}
\newcommand{\Zg}{\BZ_{\ge 0}}

\newcommand{\GL}{\mathrm{GL}}
\newcommand{\GLdagger}{\mathrm{GL}^\dagger}
\newcommand{\gl}{\frak{gl}}
\newcommand{\GO}{\mathrm{GO}}
\newcommand{\GSpin}{\mathrm{GSpin}}
\newcommand{\GU}{\mathrm{GU}}
\newcommand{\hg}{{\mathrm{hg}}}
\DeclareMathOperator{\Hom}{Hom}

\newcommand{\id}{\ensuremath{\mathrm{id}}\xspace}
\let\Im\relax
\DeclareMathOperator{\Im}{Im}
\newcommand{\Ind}{{\mathrm{Ind}}}
\newcommand{\inj}{\hookrightarrow}
\newcommand{\Int}{\ensuremath{\mathrm{Int}}\xspace}
\newcommand{\inv}{^{-1}}
\DeclareMathOperator{\Isom}{Isom}

\DeclareMathOperator{\Jac}{Jac}

\DeclareMathOperator{\Ker}{Ker}

\DeclareMathOperator{\Lie}{Lie}
\newcommand{\loc}{\ensuremath{\mathrm{loc}}\xspace}

\newcommand{\M}{\mathrm{M}}
\newcommand{\Mp}{{\mathrm{Mp}}}

\newcommand{\naive}{\ensuremath{\mathrm{naive}}\xspace}
\newcommand{\new}{{\mathrm{new}}}
\DeclareMathOperator{\Nm}{Nm}
\DeclareMathOperator{\NS}{NS}

\newcommand{\OGr}{\mathrm{OGr}}
\DeclareMathOperator{\Orb}{Orb}
\DeclareMathOperator{\ord}{ord}

\DeclareMathOperator{\proj}{proj}

\DeclareMathOperator{\rank}{rank}

\newcommand{\PGL}{{\mathrm{PGL}}}
\DeclareMathOperator{\Pic}{Pic}

\newcommand{\rc}{\ensuremath{\mathrm{rc}}\xspace}
\renewcommand{\Re}{{\mathrm{Re}}}
\newcommand{\red}{\ensuremath{\mathrm{red}}\xspace}
\newcommand{\reg}{{\mathrm{reg}}}
\DeclareMathOperator{\Res}{Res}
\newcommand{\rs}{\ensuremath{\mathrm{rs}}\xspace}

\DeclareMathOperator{\uAut}{\underline{Aut}}
\newcommand{\Sel}{{\mathrm{Sel}}}
%\newcommand{\Sha}{{\underline{\mathrm{|||}}}}
%\newcommand{\Sha}{{\hbox{\cyr Sh}}
\newcommand{\Sim}{{\mathrm{Sim}}}
\newcommand{\SL}{{\mathrm{SL}}}
\DeclareMathOperator{\Spec}{Spec}
\DeclareMathOperator{\Spf}{Spf}
\newcommand{\SO}{{\mathrm{SO}}}
\renewcommand{\O}{{\mathrm{O}}}
\newcommand{\Sp}{{\mathrm{Sp}}}
\newcommand{\SU}{{\mathrm{SU}}}
\DeclareMathOperator{\Sym}{Sym}
\DeclareMathOperator{\sgn}{sgn}

\DeclareMathOperator{\tr}{tr}

\newcommand{\U}{\mathrm{U}}
\newcommand{\ur}{{\mathrm{ur}}}

\DeclareMathOperator{\vol}{vol}



\newcommand{\CCO}{O}



\newcommand{\wt}{\widetilde}
\newcommand{\wh}{\widehat}
\newcommand{\pp}{\frac{\partial\ov\partial}{\pi i}}
\newcommand{\pair}[1]{\langle {#1} \rangle}
\newcommand{\wpair}[1]{\left\{{#1}\right\}}
\newcommand{\intn}[1]{\left( {#1} \right)}
\newcommand{\norm}[1]{\|{#1}\|}
\newcommand{\sfrac}[2]{\left( \frac {#1}{#2}\right)}
\newcommand{\ds}{\displaystyle}
\newcommand{\ov}{\overline}
\newcommand{\incl}{\hookrightarrow}
\newcommand{\lra}{\longrightarrow}
\newcommand{\imp}{\Longrightarrow}
\newcommand{\lto}{\longmapsto}
\newcommand{\bs}{\backslash}


\newcommand{\uF}{\underline{F}}
\newcommand{\ep}{\varepsilon}

%%% some additional macros


\newcommand{\nass}{\noalign{\smallskip}}
\newcommand{\htt}{h}
\newcommand{\cutter}{\medskip\medskip \hrule \medskip\medskip}


\def\tw{\tilde w}
\def\tW{\tilde W}
\def\tS{\tilde \BS}
\def\kk{\mathbf k}
\DeclareMathOperator{\supp}{supp}
% Equation  \AMSname
% Theorem   \theoremname

\def\Gu{[G_u]}  %unipotent classes in G
\def\Gue{[G_u]_e}  %unipotent classes in G, in the image W_e
\def\Gou{[G_{0,u}]}  %unipotent classes in G_0  (can't use 0 nicely in command name, use o instead)
\def\Goue{[G_{0,u}]_e}  %unipotent classes in G_0, in the image of W_e
\def\Wc{[W]}    %conjugacy classes in W
\def\Wec{[W_e]} %elliptic conjugacy classes in W
\def\WDec{[W^D_e]} %elliptic conjugacy classes in W
\def\tPminus{\wt{\CP_{-1}}}
\def\tPplus{\wt{\CP_{1}}}
\def\epsilonmax{\epsilon_{\text{max}}}
\def\leW{\preceq_W}
\def\leu{\preceq_u}
\def\Ftwo{\overline{\BF_2}}
% Theorem environments.
%
\newtheorem{theorem}{Theorem}
\newtheorem{proposition}[theorem]{Proposition}
\newtheorem{lemma}[theorem]{Lemma}
\newtheorem {conjecture}[theorem]{Conjecture}
\newtheorem{corollary}[theorem]{Corollary}
\newtheorem{axiom}[theorem]{Axiom}
\theoremstyle{definition}
\newtheorem{definition}[theorem]{Definition}
\newtheorem{example}[theorem]{Example}
\newtheorem{exercise}[theorem]{Exercise}
\newtheorem{situation}[theorem]{Situation}
\newtheorem{remark}[theorem]{Remark}
\newtheorem{remarks}[theorem]{Remarks}
\newtheorem{question}[theorem]{Question}




\numberwithin{equation}{section}
\numberwithin{theorem}{section}






%%%% macros added by Brian
%%%% many of these require the etoolbox package, which should be loaded above

\newcommand{\aform}{\ensuremath{\langle\text{~,~}\rangle}\xspace}
\newcommand{\sform}{\ensuremath{(\text{~,~})}\xspace}

\newcounter{filler}

% gets rid of indentation in itemize and enumerate enivronments, and adds
% a small space between list items:
\setitemize[0]{leftmargin=*,itemsep=\the\smallskipamount}
\setenumerate[0]{leftmargin=*,itemsep=\the\smallskipamount}

% basic right arrow, short in inlines and long in displays
\renewcommand{\to}{%
   \ifbool{@display}{\longrightarrow}{\rightarrow}%
   }
% redefine \mapsto to be short in inlines and long in displays
\let\shortmapsto\mapsto
\renewcommand{\mapsto}{%
   \ifbool{@display}{\longmapsto}{\shortmapsto}%
   }
% stretchable labeled right (2nd is xy-style) & left arrows, well-behaved inline or displayed
\newlength{\olen}
\newlength{\ulen}
\newlength{\xlen}
\newcommand{\xra}[2][]{%
   \ifbool{@display}%
      {\settowidth{\olen}{$\overset{#2}{\longrightarrow}$}%
       \settowidth{\ulen}{$\underset{#1}{\longrightarrow}$}%
       \settowidth{\xlen}{$\xrightarrow[#1]{#2}$}%
       \ifdimgreater{\olen}{\xlen}%
          {\underset{#1}{\overset{#2}{\longrightarrow}}}%
          {\ifdimgreater{\ulen}{\xlen}%
             {\underset{#1}{\overset{#2}{\longrightarrow}}}
             {\xrightarrow[#1]{#2}}}}%
      {\xrightarrow[#1]{#2}}
   }
\makeatother
\newcommand{\xyra}[2][]{%
   \settowidth{\xlen}{$\xrightarrow[#1]{#2}$}%
   \ifbool{@display}%
      {\settowidth{\olen}{$\overset{#2}{\longrightarrow}$}%
       \settowidth{\ulen}{$\underset{#1}{\longrightarrow}$}%
       \ifdimgreater{\olen}{\xlen}%
          {\mathrel{\xymatrix@M=.12ex@C=3.2ex{\ar[r]^-{#2}_-{#1} &}}}%
          {\ifdimgreater{\ulen}{\xlen}%
             {\mathrel{\xymatrix@M=.12ex@C=3.2ex{\ar[r]^-{#2}_-{#1} &}}}
             {\mathrel{\xymatrix@M=.12ex@C=\the\xlen{\ar[r]^-{#2}_-{#1} &}}}}}%
      {\mathrel{\xymatrix@M=.12ex@C=\the\xlen{\ar[r]^-{#2}_-{#1} &}}}%
   }
\makeatletter
\newcommand{\xla}[2][]{%
   \ifbool{@display}%
      {\settowidth{\olen}{$\overset{#2}{\longleftarrow}$}%
       \settowidth{\ulen}{$\underset{#1}{\longleftarrow}$}%
       \settowidth{\xlen}{$\xleftarrow[#1]{#2}$}%
       \ifdimgreater{\olen}{\xlen}%
          {\underset{#1}{\overset{#2}{\longleftarrow}}}%
          {\ifdimgreater{\ulen}{\xlen}%
             {\underset{#1}{\overset{#2}{\longleftarrow}}}
             {\xleftarrow[#1]{#2}}}}%
      {\xleftarrow[#1]{#2}}
   }
% isomorphism arrow, short in inlines and long in displays
\newcommand{\isoarrow}{%
   \ifbool{@display}{\overset{\sim}{\longrightarrow}}{\xrightarrow\sim}%
   }




\begin{document}
\thispagestyle{plain}

\author{Jeffrey Adams}
\author{Xuhua He}
\author{Sian Nie}

%use one or the other of these:

% using \documentclass[a4paper, reqno, 10pt]{article}:
%\affil{Department of Mathematics, University of Maryland, jda@math.umd.edu}
%\affil{Department of Mathematics, University of Maryland, xuhuahe@math.umd.edu}
%\affil{Institute of Mathematics, Academy of Mathematics and Systems Science, Chinese Academy of Sciences, 100190, Beijing, China
%  , niesian@amss.ac.cn}

% %\documentclass[a4paper, reqno, 12pt]{amsart}
\address{Department of Mathematics, University of Maryland, jda@math.umd.edu}
\address{Department of Mathematics, University of Maryland, xuhuahe@math.umd.edu}
\address{Institute of Mathematics, Academy of Mathematics and Systems Science, Chinese Academy of Sciences, 100190, Beijing, China
  , niesian@amss.ac.cn}


%\date{}                     %% if you don't need date to appear


\title[Partial orders]{Partial orders on conjugacy classes in the Weyl group and on unipotent conjugacy classes}

%\thanks{}

%\keywords{}
%\subjclass[2010]{}

\date{\today}

\maketitle

\tableofcontents

\section*{Introduction}

This is the missing introduction.

\subsection{Lusztig's map}
Let $G$ be a connected reductive group over an algebraically closed
field $\BF$ and let $W$ be the Weyl group of $G$. Let $\Gu$ be the set of
unipotent conjugacy classes in $G$ and $\Wc$ be the set of conjugacy
classes of $W$. Lusztig has defined a surjective map
$\Phi: \Wc \to \Gu$ (\cite[Theorem 0.4]{L1}).
This construction is generalized to twisted conjugacy classes in \cite{L3}.

Roughly speaking, the map $\Phi$ is constructed as follows. For the introduction
 we assume $\BF$ has characteristic $0$.
Let
$\CC \in \Wc$ and $w \in \CC_{\min}$ be a minimal length element of
$\CC$. We look at the intersection of Bruhat double coset $B w B$ with
unipotent conjugacy classes and we select the minimal
unipotent class which gives a nonempty intersection. It is pointed out
in \cite[\S 0.1]{L1} that ``the fact that the procedure actually works
is miraculous''.  We are interested in understanding the map $\Phi$ is
a more conceptual way.

In this paper we are concerned with the elliptic conjugacy classes
$\Wec$ of $W$.  The restriction of $\Phi$ to $\Wec$ is  injective,
and the image contains all distinguished unipotent
conjugacy classes (\cite[Proposition 0.6]{L1}).


The set $\Gu$ has the natural partial ordering by closure
relations, denote $\leu$. On the other hand the
second-named author \cite{He07} (see also \cite[\S
1.10.3]{He-CDM}) has introduced a partial order on $\Wec$,
induced from the Bruhat order on  minimal length elements of the
elliptic conjugacy classes of $W$, which we denote $\leW$.

It is a natural question to consider how Lusztig's map
behaves with respect to these partial orders.
Dudas, Michel and the second-named author
\cite{DHM} conjectured that $\Phi$ gives an order-reversing bijection
from $\Wec$ to $\Phi(\Wec) \subset \Gu$.

In \cite[Conjecture 3.7]{DM}, Dudas and Malle conjectured that a
similar result holds for twisted type $A$. In \cite[Proposition 3.11
\& 3.14]{DM} they verified the conjecture for some special family of
twisted elliptic conjugacy classes of type $A$. It is also verified in
\cite{DM} by computer that the conjecture holds for twisted $A_n$ with
$n \le 10$. The compatibility of the partial orders is used in
\cite[\S 5]{DM} to study the decomposition numbers of the unipotent
$l$-blocks of finite unitary groups.

Our first main result is  that $\Phi$ is order reversing in a strong sense.

\begin{theorem}
\label{t:main}
Let $\CC, \CC' \in \Wec$. Then $\CC \leW\CC'$ if and only if
$\Phi(\CC') \leu\Phi(\CC)$.
\end{theorem}

This holds also in the twisted setting.

\bigskip

\subsection{Lusztig's map for classical groups}
We now consider the question of finding a simpler description of
Lusztig's map $\Phi$ for classical groups. Note that $\Phi$
takes $\Wc$ to $\Gu$, and although $\Wc$ is independent of the field
$\Gu$ is not. For classical groups the most interesting case is when $\BF$
has characteristic $2$.
Let $G_p$ be a classical group over a field of characteristic $p$,
and let $G_0$ be the group over $\C$ with the same root datum as $G$.
There is a natural
injective map $\pi:[G_{0,u}]\hookrightarrow [G_{p, u}]$ which is an
isomorphism of partially ordered sets from $[G_{0,u}]$ to its image. This map is bijective if $p \neq 2$.
Hence the case of characteristic $2$ has the most information, and it
turns out to be the simplest case as well.
So we now restrict to the case $p=2$.

For this introduction we consider
$\GL(n), \SO(n), \O(n), \Sp(2n)$, see Section ? for the case of
twisted type $A$.  Each group comes with its tautological embedding
$\iota:G\rightarrow G^*$ where $G^*=GL(m)$ ($m=n$ or $2n$).
This induces
a homomorphism $\iota:W\rightarrow W^*$ where $W^*$ is the Weyl group
of $G^*$.

Consider the following diagram:

$$
\xymatrix{
[W^*]\ar[r]^{\Phi^*}&[G^*_u]\ar[d]^\Gamma\\
\Wec\ar[r]^{\Phi}\ar[u]^{\iota}&\Gu\\
}
$$

\medskip
The left vertical arrow is induced by  the homomorphism $\iota:W\rightarrow W^*$.
The top row is Lusztig's map for $G^*=GL(m)$, which is elementary:
both $\Wc$ and $\Gu$ are parametrized by partitions of $m$, and $\Phi^*$ corresponds to the identity map on partitions.
The bottom row is Lusztig's map for $G$.

Suppose $\CU\in[G^*_u]$.
Then (see Section ?)  $\overline{\CU}\cap G$
is the closure of a single unipotent orbit for $G$. Hence
there is a well defined map $\Gamma:[G^*_u]\rightarrow\Gu$ satisfying:
$$
\overline{\CU}\cap G=\overline{\Gamma(\CU)}\quad (\CU\in[G^*_u]).
$$

\begin{theorem}
\label{t:second}
Suppose $G=GL(n), SO(n),O(n)$ or $Sp(2n)$ and $p=2$.
Then the preceding diagram commutes:

$$
\Phi(\CC)=\Gamma(\Phi^*(\iota(\CC))).
$$
\end{theorem}

A twisted version of this holds as well.

\subsection{A similar phenomenon for affine Weyl groups}

Before we discuss the strategy towards Theorem \ref{t:main}, we make a
short digression and discuss a similar phenomenon.

For simplicity, we only discuss  split groups here but the result
holds in general.

Let $G$ be a connected reductive group. The Frobenius morphism $\s$ of
$\overline{\mathbb F_q}$ over $\mathbb F_q$ induces a Frobenius
morphism $\s$ on $G(\overline{\mathbb F_q}((t)))$. Let $B(G)$ be the
set of $\s$-twisted conjugacy classes on
$G(\overline{\mathbb F_q}((t)))$. Let $\tW$ be the Iwahori-Weyl group
and $\underline \tW$ be the set of conjugacy classes of $\tW$. Inside
$[\tW]$, there is a special subset, the set of straight
conjugacy classes, which we denote by $[\tW_{str}]$. It is
proved in \cite{He-Ann} that there is a natural bijection
$\tilde \Phi: [\tW_{str}] \to B(G)$, which is induced from
any lifting $\tW \to G(\overline{\mathbb F_q}((t)))$.

The closure relation gives the partial order on $B(G)$. There is also
a partial order on $[\tW_{str}]$ induced from the Bruhat
order on $\tW$, similar to the definition of the partial order
$\leW$ on $\Wec$ we discussed earlier. It is proved in
\cite[Theorem B]{He-KR} that these two partial orders coincide via the
bijection $\tilde \Phi: [\tW_{str}] \to B(G)$.

The proof uses the reduction method of Deligne and Lusztig \cite{DL},
some remarkable combinatorial properties on the straight conjugacy
classes \cite{HN2} and a deep result in arithmetic geometry, the
purity theorem for the Newton stratification associated with
$F$-crystal obtained by de Jong-Oort \cite{JO}, Hartl-Viehmann
\cite{HV}, Viehmann \cite{Vi} and Hamacher \cite{Ha}.

\subsection{Difference between the  finite and affine cases}

Although in both finite and affine cases, we compare the partial
orders arising from combinatorics and from geometry, there are some
essential differences between the two cases we discussed above.

First, in the affine case, we do not consider the conjugation action
on the loop groups, but the Frobenius-twisted conjugacy classes
instead. Although the study of the Frobenius-twisted conjugacy classes
are quite involved, it is, in some sense, simpler than the unipotent
conjugacy classes. Second, the construction of the map from conjugacy
classes of Weyl groups to the (twisted) conjugacy classes of reductive
groups in the finite and affine case are quite different. In the
affine case, the map is induced from any lifting
$\tW \to G(\overline{\mathbb F_q}((t)))$, while in the finite case the
construction is rather a miracle. Finally, in the affine case, the map
preserves the partial orders, while in the finite case, as we show in
this paper, the map reverses the partial orders.

\subsection{The strategy}

Now we discuss the strategy towards the proof of Theorem
\ref{t:main}. For exceptional groups, we verify the statement by
computer in \S?. For classical groups, the elliptic conjugacy classes
and the unipotent classes are parametrized by certain partitions of
different integers. For example, in type $B_n$, the elliptic conjugacy
classes correspond to partitions of $n$ while the unipotent conjugacy
classes correspond to certain partitions of $2n+1$. We first show that
the partial orders on the two sets of partitions are compatible. The
main difficulty is to verify that the partial order on the elliptic
conjugacy classes coincide with the partial order on the corresponding
partitions. We use the explicit description of the Bruhat order for
the Weyl groups of classical type to relate the partial order
$\leW$ on $\Wec$ to the natural partial order on the
partitions. It is known that the unipotent conjugacy classes for
classical groups are parametrized by some partitions.

\section{Main result}

\subsection{Preliminary} Let $G$ be an affine algebraic group over an algebraically closed field of characteristic $p \ge 0$ such that the identity component $G^0$ of $G$ is reductive. Let $T$ be a maximal torus of $G^0$ and $B \supset T$ be a Borel subgroup of $G^0$. Let $W^0=N_{G^0}(T)/T$ be the Weyl group of $G^0$ and $W=N_G(T)/T$ be the (extended) Weyl group of $G$. The length function $\ell$ on $W^0$ extends in a unique way to a length function on $W$, which we still denote by $\ell$. Let $S \subset W^0$ be the set of simple reflections.

Let $D$ be a connected component of $G$ and $W^D=(N_G(T) \cap D)/T$ be a left/right $W^0$-coset of $W$. As in \cite[\S 1.1]{L3}, $D$ defines a group automorphism $\e_D: W \to W$ which preserves the length. Let $[W]$ be the set of conjugacy classes of $W$ and $[W^D]$ be the set of $W^0$-conjugacy classes of $W$ that intersects $D$. An element $w \in W^D$ (or its conjugacy class $C$ in  $W$) is said to be {\it elliptic} if for any $J \subsetneqq S$ with $\e_D(J)=J$, we have $C \cap W_J=\emptyset$. Let $\WDec$ the set of elliptic $W^0$-conjugacy classes of $W$ that intersects $W^D$.

From now on, we assume that $D$ contains an unipotent element of $G$. Let $[D_u]$ be the set of $G^0$-conjugacy classes of $D$ which are unipotent. In \cite{L1} and \cite{L3}, Lusztig introduced a ``miraculous'' map $\Phi: [W^D] \to [D_u]$. It is proved in loc. cit. that 

\begin{itemize}
\item The map $\Phi: [W^D] \to [D_u]$ is surjective. 

\item The restriction to elliptic conjugacy classes $\Phi_e: \WDec \to [D_u]$ is injective. 
\end{itemize}

\subsection{Partial orders} We define a partial order on unipotent classes by the closure relations as usual:
$\CC\leu \CC'$ if $\CC\subset \overline{\CC'}$.

Now we recall the partial order on $\WDec$ introduced in \cite[\S 4.7]{He07}. Let $\CC, \CC' \in \WDec$. We denote by $\CC_{\min}$ (respectively $\CC'_{\min}$) the set of minimal length elements in $\CC$ (respectively $\CC'$). Then the following conditions are equivalent:

\begin{enumerate}
  \item For some $w \in \CC_{\min}$, there exists $w' \in \CC'_{\min}$ such that $w' \le w$;
  \item For any $w \in \CC_{\min}$, there exists $w' \in \CC'_{\min}$ such that $w' \le w$.
\end{enumerate}

If these conditions are satisfied, then we write $\CC' \leW \CC$. By the equivalence of the conditions (1) and (2) above, the relation $\leW$ is transitive. This gives a natural partial order on the set $\WDec$.

By \cite[Corollary 4.5]{He07}, $\CC_{\min}$ is the set of minimal elements in $\CC'$ with respect to the Bruhat order of $W$. Thus the conditions (1) and (2) above are also equivalent to the following conditions: 

\begin{enumerate}
\setcounter{enumi}{2}
  \item For some $w \in \CC_{\min}$, there exists $w' \in \CC'$ such that $w' \le w$;
  \item For any $w \in \CC_{\min}$, there exists $w' \in \CC'$ such that $w' \le w$.
\end{enumerate}

The condition (3) will be used to study the partial order on $\WDec$ for exceptional groups. 

Now we state the main theorem of the paper. 

\begin{theorem}\label{main}
The map $\Phi_e: \WDec \to [D_u]^{op}$ gives a bijection from the poset $\WDec$ to its image. Here $[D_u]^{op}$ is the same as $[D_u]$ as sets, but with reversed partial order. 

In other words, let $\CC, \CC' \in \WDec$. Then $\CC' \leW \CC$ if and only if $\Phi(\CC) \leu \Phi(\CC')$.
\end{theorem}

\subsection{Reduction to almost simple groups} In this subsection, we show that to prove Theorem \ref{main}, it suffices to consider the case where $G^0$ is simple. The reduction procedure is the similar to \cite[\S 1.5--1.11]{L3}.

First, we may replace $G$ by the subgroup generated by $D$. Let $G'=G/Z(G^0)$ and $\pi: G \to G'$ be projection map. As the Weyl groups of $G$ and $G'$ are naturally identical, Theorem \ref{main} holds for $(G, D)$ if and only if it holds for $(G', D')$, where $D'=\pi(D)$. 

Now we may assume that $G^0$ is semisimple and simply connected. We write $G^0$ as $G^0=G_1 \times \ldots \times G_k$, where each $G_i \neq \{1\}$ is a minimal closed connected normal subgroup of $G$. For any $i$, let $G'_i=G/(G_1 \times \ldots \times \hat G_i \times \ldots \times G_k)$ and $D_i$ be the image of $D$ in $G'_i$. Let $G'=G'_1 \times \ldots \times G'_k$. We may then identify $G$ with a closed subgroup of $G'$ with the same identity component. Under this identification, $D$ becomes $D_1 \times \ldots \times D_k$. Let $W'$ be the extended Weyl group of $G'$ and $W'_i$ be the extended Weyl group of $G'_i$. Then $W'=W'_1 \times \ldots \times W'_k$ and we may identify $W^D$ with $(W'_1)^{D_1} \times \ldots \times (W'_k)^{D_k}$. Under this identification, $$\WDec=[(W'_1)^{D_1}_e] \times \ldots \times [(W'_k)^{D_k}_e]$$ and $\leW$ on $\WDec$ coincides with $\preceq_{W, 1} \times \ldots \times \preceq_{W, k}$ on $[(W'_1)^{D_1}_e] \times \ldots \times [(W'_k)^{D_k}_e]$. By \cite[\S 1.8]{L3}, $\Phi_e$ on $\WDec$ coincides with $\Phi_{e, 1} \times \ldots \times \Phi_{e, k}$ on $[(W'_1)^{D_1}_e] \times \ldots \times [(W'_k)^{D_k}_e]$. Thus if Theorem \ref{main} holds for each $(G'_i, D_i)$, then it holds for $(G, D)$. 

Now we may assume that $G^0$ is semisimple, simply connected and that $G$ has no nontrivial closed connected normal subgroups. By \cite[\S 1.9]{L3}, $G^0=H_1 \times \ldots \times H_{m}$, where $H_i$ are connected, simply connected, almost simple, closed subgroups of $G^0$ and there exists a $c \in D$ such that $H_i=c^i H_1 c^{-i}$ for $0 \le i \le m-1$ and $c^m H_1 c^{-m}=H_1$. Let $G'$ be the subgroup of $G$ generated by $H_1$ and $c^m$ and $D'=c^m H_0$ be a connected component of $G'$. By \cite[\S 1.9]{L3}, we may identify $[D_u]$ with $[D'_u]$ and $[W^D]$ with $[(W')^{D'}]$ and under this identification, we have the following commutative diagram
\[
\xymatrix{
[(W')^{D'}] \ar@{=}[r] \ar[d]_-{\Phi'} & [W^D] \ar[d]^-{\Phi} \\
[D'_u] \ar@{=}[r] & [D_u].
}
\]
Thus Theorem \ref{main} holds for $(G, D)$ if and only if it holds for $(G', D')$. 

Therefore, to prove Theorem \ref{main}, it suffices to consider the cases where $G^0$ is almost simple. 

\section{Unipotent conjugacy classes of classical groups}
\label{s:classical}

In this section, we recollect some facts on the unipotent conjugacy
classes of classical groups, over an
algebraically closed field $\BF$ of any characteristic. We follow \cite{Spa}*{Section I.2}.

We consider the groups $\GL(n), \Sp(2n), SO(n)$ and $\O(n)$.

Suppose $V$ is a finite-dimensional vector
space over $\BF$.
We define a disconnected group containing $\GL(V)$ $(\dim(V)\ge 3)$ as in
\cite{Spa}*{Section I.2.7}.
Define $\GLdagger(V)=G_0\cup G_1$ where $G_0=GL(V)$ and $G_1$ is
the set of non-singular bilinear forms $\phi:V\times V\rightarrow \BF$.
We define the product structure on $G$ as follows. The product on
$G_0$ is the usual one. If $g\in G_0,\phi\in G_1$ then $(g\phi)(v,w)=\phi(g\inv v,w)$
and $(\phi g)(v,w)=\phi(v,gw)$. If $\phi,\psi\in G_1$ then $\phi\psi$ is the unique element of $G_0$ satisfying
$\phi((\phi\psi)v,w)=\psi(w,v)$.

It is easy to see $\GLdagger(V)$ is a group and
$\GLdagger(V)/\GL(V)\simeq\Ztwo$.  Also $\phi$ is symmetric if and
only if $\phi^2=1$, so there is a unique $\GL(V)$-conjugacy class of
such elements. Choose a basis of $V$ and identify $\GL(V)$ with
$\GL(n)$, and set $\delta(v,w)=v\cdot w$. Then $\delta^2=1$ and
$\delta g\delta=\,^tg\inv$ for $g\in \GL(V)$.

By a {\it classical group} we mean one of the groups $\GL(n), \Sp(2n),\SO(n), \O(n)$ or $\GLdagger(n)$.
The groups $\GL(n)$ and $\Sp(2n)$ are connected.
The identity component of $\O(n)$ is $\SO(n)$, and $\O(n)/\SO(n)$ is trivial if $p=2$, and has order $2$ otherwise. \remind{for odd $n$?}

Let $\CP(n)$ be the set of partitions of $n$. We write a partition of $n$ as
$\alpha=(\alpha_1,\dots, \alpha_\ell)$ with $\alpha_1\ge \dots\ge \alpha_\ell>0$ and $\sum\alpha_i=n$.
On occasion we will allow $\alpha_{\ell-1}>\alpha_\ell=0$.
We define the standard partial order on partitions of the same integer $n$: $\alpha\le\beta$ if for all $k$,
$\sum_{i=1}^k\alpha_i=\sum_{i=1}^k\beta_k$.

We identify partitions with Young diagrams, and define the transpose
partition as usual. If $\alpha=(\alpha_1,\dots,\alpha_\ell)$ is a
partition we let $\alpha^*=(\alpha_1^*,\dots, \alpha_m^*)$ be the
transpose partition. In particular $\alpha_1^*$ is the number of rows
of $\alpha$.

Given a partition $\alpha$ define the multiplicity function $c_\alpha:\Zg\rightarrow \Zg$ as usual:
$c_\alpha(k)=|\{j\mid \alpha_j=k\}|$. In particular $c_\alpha(k)=0$ for $k=0$ or $k>\alpha_1$.
For $\k=\pm 1$ let
$$
\CP_\k(n)=\{\alpha \in \CP(n)\mid c_\alpha(i)\text{ is even if }  (-1)^i=\kappa\}
$$
Let $\CP(n)_0$ be the partitions of $n$ with an even number of parts,
and let $\CP(n)^{\text{odd}}$  be the partitions consisting only of odd parts.

\subsection{Unipotent classes in good characteristic}
\label{s:good}


If $G=\GLdagger(n), \Sp(2n)$ or $\O(n)$ then the characteristic $p=2$ is said to be {\it bad} (for $G$).
Otherwise, including all groups in characteristic $0$, the characteristic is said to be {\it good}.

Suppose the characteristic of $\BF$ is good. Then every unipotent
element of $G$ is contained in $G^0$, and
the unipotent classes are parametrized as follows.

\begin{enumerate}
\item $\GL(n)$ or $\GLdagger(n)$: $\CP(n)$
\item $\O(2n+1)$: $\CP_1(2n+1)$
\item $\Sp(2n)$: $\CP_{-1}(2n)$
\item $\O(2n)$: $\CP_1(2n)$
\end{enumerate}

We also consider the unipotent conjugacy classes of $\SO(n)$.  If $n$
is odd the unipotent conjugacy class of $\SO(n)$ and $O(n)$ are in
bijection, and the same holds in for $\SO(2n)$ if $n$ is odd.  If $n$
is even the unipotent $\SO(2n)$-conjugacy classes in $\SO(2n)$ are parametrized
by $\CP_1(2n)$, except that every partition with only even parts
corresponds to two classess; there are $p(n/2)$ of these classes where
$p$ is the partition function. We will not need to distinguish between
these two conjugacy classes. \remind{?}

We write $\CU_{\alpha}$ for the unipotent class parametrized by a partition $\alpha$.
Then the partial order on unipotent conjugacy classes is given by the partial order on partitions:
$\CU_{\a} \leu \CU_{\b}$ if and only if $\a \le \b$.

\subsection{Unipotent conjugacy classes in bad characteristic}
\label{s:bad}

Suppose $G$ is a classical group and the characteristic $p$ of $\BF$ is
bad, i.e. $p=2$.  There is a bijective algebraic group homomorphism
from $\Sp(2n)$ to $SO(2n+1)$ (although the inverse is not algebraic),
which induces a bijection of unipotent classes. Also
$SO(2n+1)=O(2n+1)$ so we do not need to consider these groups.

We first treat the cases $G=Sp(2n)$  and $O(2n)$.

Consider a set $\{\omega,0,1\}$ where $\omega$ is a formal element,
satisfying $\omega<0<1$.  For $n\ge 1$ define
$\tPminus(n)$ to be the set of pairs $(\alpha,\epsilon)$, where
\begin{enumerate}
\item $\alpha\in \CP_{-1}(2n)$ (i.e. odd rows have even multiplicity);
\item $\epsilon:\Zg\rightarrow\{\omega,0,1\}$.
\end{enumerate}
The function $\epsilon$ is required to satisfy, for all $i\ge 0$:
$$
\epsilon(i)=
\begin{cases}
1,& \text{ if } i=0, G=Sp(n);\\
0,& \text{ if } i=0, G=O(n);\\
\omega,& \text{ if } i\text{ odd};\\
\omega,& \text{ if } i>0,\,c_\alpha(i)=0;\\
1,& \text{ if } i>0\text{ even},\, c_\alpha(i)\text{ odd};\\
0\text{ or }1,& \text{ if } i>0\text{ even},\, c_\alpha(i)>0 \text{ even}.,
\end{cases}
$$
Note that $\tPminus(n)$ is empty if $n$ is odd, and $\epsilon(i)$ is determined by $\alpha$ except for even rows of even multiplicity.

\begin{proposition}
  If $p=2$ the unipotent conjugacy classes in $Sp(2n)$ or $O(2n)$ are
  in bijection with $\tPminus(2n)$.
\end{proposition}

We write $\CU_{\alpha,\epsilon}$ for the unipotent class parametrized by $(\alpha,\epsilon)$.

The map from unipotent classes to $\tPminus(n)$ is defined as
follows.  First assume $G=Sp(2n)$, and let $\langle\,,\,\rangle$ be
the symplectic form defining $G$.  We embed
$\phi:\Sp(2n)\rightarrow G^*=GL(2n)$ as usual.  If $g\in G$ is
unipotent then $\phi(g)\in G^*$ is unipotent, and so
corresponds to a partition  $\alpha$ of $2n$; it is easy to see $\alpha\in\CP_{-1}(2n)$.

Suppose $i>0$ is even and $c_\alpha(i)>0$. Set $\epsilon(i)=0$ if
$\langle (g-1)^{i-1}v,v\rangle=0$ for all $v\in \text{ker}(g-1)^i$, and $\epsilon(i)=1$ otherwise.
Together with the conditions above this defines $\epsilon$ uniquely.

Next, if $G=O(2n)$ we note that every unipotent conjugacy class in
$Sp(2n)$ intersects $O(2n)$ in a unique conjugacy class, and this
defines a bijection between unipotent conjugacy classes in $Sp(2n)$ and $O(2n)$.


Write $\CU_{\alpha,\epsilon}$ for
the unipotent conjugacy class associated to
$(\alpha,\epsilon)\in \tPminus(2n)$.


In the case of $G=O(2n)$ we need to distinguish between those
$G$-conjugacy classes contained in $\SO(2n)$ and those which are not.

Define $\tPminus(2n)_0\subset \tPminus(2n)$ to be the pairs $(\alpha,\epsilon)$ such that $\alpha_1^*$ is even, and
set $\tPminus(2n)_1=\tPminus(2n)\backslash \tPminus(2n)_0$.

\begin{lemma}
Suppose $(\alpha,\epsilon)\in \tPminus(2n)$.
Then $\CU_{\alpha,\epsilon}\subset \SO(2n)$ if and only if $(\alpha,\epsilon)\in\tPminus(2n)_0$.
\end{lemma}

Thus $\tPminus(2n)_0$ (respectively $\tPminus(2n)_1$) is in bijection with the unipotent  $\O(2n)$ conjugacy classes in $\SO(2n)$ (respectively
$\O(2n)\backslash\SO(2n)$).

Finally we consider
unipotent $\SO(2n)$-conjugacy classes.
If $(\alpha,\epsilon)\in \tPminus(2n)_0$ then
 $\CU_{\alpha,\epsilon}\subset \SO(2n)$ is the union of two $\SO(2n)$-conjugacy classes if
 for all $i$, $\alpha_i$ and $c_\alpha(i)$ are even and $\epsilon(i)=0$. Otherwise
 $\CU_{\alpha,\epsilon}\subset\SO(2n)$ is a single $\SO(2n)$-conjugacy class.
 Again we will not need to specify one of these two orbits more precisely. \remind{?}

Recall every characteristic for $\GL(n)$ is good, and the unipotent
classes for $\GL(n)$ are parametrized by partitions of $n$. Now we
consider $\GLdagger(n)$ ($n\ge 3$). Recall we write
$\GLdagger(n)=G_0\cup G_1$ where $G_0=\GL(n)$ and $G_1$ is the set of
non-singular bilinear forms.

\begin{lemma}
The unipotent conjugacy classes of $\GLdagger(n)$ which are contained in
$\GL(n)$ are in bijection with the unipotent conjugacy classes of $\GL(n)$.
\end{lemma}

In other words if $\CU\subset \GL(n)$ is a unipotent conjugacy class for $\GLdagger(n)$
then it is a single $\GL(n)$-orbit.

So consider the unipotent conjugacy classes of $\GLdagger(n)$ in $\GLdagger(n)\backslash \GL(n)$.

Define $\tPplus(n)$ to be the set of pairs $(\alpha,\epsilon)$,
where
\begin{enumerate}
\item $\alpha\in \CP_{1}(n)$ (i.e. even rows have even multiplicity);
\item $\epsilon:\Zg\rightarrow\{\omega,0,1\}$.
\end{enumerate}
The function $\epsilon$ is required to satisfy, for all $i\ge 0$:
$$
\epsilon(i)=
\begin{cases}
\omega,& \text{ if } i\text{ even};\\
\omega,& \text{ if } c_\alpha(i)=0;\\
1,& \text{ if } i\text{ odd}, c_\alpha(i)\text{ odd};\\
0\text{ or }1,& \text{ if } i\text{ odd},\, c_\alpha(i)>0\text{ even}.
\end{cases}
$$

\begin{proposition}
If $p=2$ the unipotent conjugacy classes of $\GLdagger(n)$ in $\GLdagger(n)\backslash \GL(n)$ are parametrized by $\tPplus(n)$.
\end{proposition}

We write $\CU_{\alpha,\epsilon}$ for the unipotent class parametrized by $(\alpha,\epsilon)$.

The map from unipotent classes in $G_1$ to parameters is defined as
follows.  Define the map $S:G_1\rightarrow G_0$ to be
$S(\phi)=\phi^2$. If $\phi$ is unipotent then so is $S(\phi)$, and
therefore $S(\phi)$ defines a partition $\alpha$ of $n$, and it is easy to see $\alpha\in\CP_1(n)$.
Suppose $i$ is odd and $c_\alpha(i)>0$ is even. Then define
$\epsilon_i=0$ if $f(v,(g-1)^{i-1}v)=0$ for all $v\in \text{ker}(g-1)^i$, and $\epsilon_i=1$ otherwise.

\subsection{Closure Relations}
\label{s:closure}

We now desribe the closure relations on nilpotent classes. As discussed
in Section \ref{s:good} if the characteristic is good then the closure
relations on classes are given by the order relation on partitions.
So assume $p=2$ and $G=\GLdagger$, $Sp(2n)$ or $O(2n)$.

We define a partial order on the sets
$\tPminus(2n)$ and $\tPplus(n)$ defined in the previous section.
Suppose $(\alpha,\epsilon)$ and $(\beta,\delta)$ are elements of one of these sets.
We say $(\alpha,\epsilon)\le (\beta,\delta)$ if

\begin{enumerate}
\item $\alpha\le \beta$ \quad $(\Leftrightarrow \alpha^*\ge \beta^*)$;
\end{enumerate}
and for all $k\ge 1$:
\begin{enumerate}
\item[(2)] $\left(\sum_{i=1}^k\beta_i^*\right)-\text{max}(\delta_k,0)\le\left(\sum_{i=1}^{k}\alpha_i^*\right)-\text{max}(\epsilon_k,0)$;
\item[(3)] If $\sum_{i=1}^k\alpha_i^*=\sum_{i=1}^k\beta_i^*$ and $\alpha^*_{k+1}-\beta^*_{k+1}$ is odd then $\delta_k\ne 0$.
\end{enumerate}



\begin{proposition}
  \label{p:order}
Suppose $p=2$ and $G=\GLdagger(n), Sp(2n)$ or $O(2n)$.
Suppose $(\alpha,\epsilon),(\beta,\delta)$ are  both in
$\tPplus(n), \tPminus(2n)$, or $\tPminus(2n)$, respectively.
In the case of $O(2n)$ assume they are both in $\tPminus(2n)_0$ or $\tPminus(2n)_1$.

Then
$\CU_{\alpha,\epsilon}\leu \CU_{\beta,\delta}$ if and
only if $(\alpha,\epsilon)\le (\beta,\delta)$.
\end{proposition}


\begin{corollary}
Fix a partition $\alpha$ of the appropriate type for $G$. Then the set $\{\CU_{\alpha,\epsilon}\}$ (as $\epsilon$ varies)
has a unique maximal element.
\end{corollary}


\begin{proof}
The choices of $\epsilon$ are given by certain choices of $0,1$; 
in each  case choose $1$.


Explicitly, suppose $\alpha\in\tPplus(n)$. Then $\epsilon(i)$ is determined unless
if $i$ is odd and $c_\alpha(i)>0$ is even, in which case it can be $0,1$; 
define $\epsilonmax^\alpha(i)=1$ for all such $i$.

Similary if $\alpha\in\tPminus(2n)$ define $\epsilonmax^\alpha(i)=1$ for all $i>0$ even such that $c_\alpha(i)>0$ is even. 
This determines $\epsilonmax^\alpha$ completely, and the result is immediate from Proposition \ref{p:order}.
\end{proof}

\begin{comment}
\section{Reduction to characteristic 2}
\label{s:p=2}

In this section $G$ is a connected reductive group, defined over an
algebraically closed field $\BF$ of characteristic $p>0$.  Let $G_0$
be the complex group with the same root datum as $G$.  Consider the
sets $\Gu$ (respectively $\Gou$) of unipotent conjugacy classes of $G$
and $G_0$, respectively.

\begin{proposition}{\cite{Spa}*{Theorem III.5.2}}
\label{p:p=2}
  There is an injective, dimension preserving map $\pi:\Gou\rightarrow \Gu$, such that $\pi$ is a isomorphism of partially ordered
  sets from $\Gou$ to its image.
\end{proposition}

\begin{corollary}
  \label{c:reductionmain}
  It is enough to prove Theorem \ref{t:main} in characteristic $0$.
\end{corollary}

We now suppose $G$ is a classical group and address the question of
reducing Theorem \ref{t:second} to characteristic $2$.  The sets $\Gu$
and $\Gou$ were described explicitly in Section \ref{s:classical}. The
map $\pi$ is given explicitly in these terms in \cite{Spa}*{Section
  III.6--8}. In each case there is a description of the image of
$\pi$, and an explicit inverse $\rho$ defined on this image.

\begin{equation}
\label{e:bijection}
\xymatrix{
\Gou \ar@/^/[rr]|\pi&&\text{Im}(\pi)\ar@/^/[ll]|\rho\ar@{^{(}->}[r]&\Gu
}
\end{equation}



Here are the sets parametrizing the objects in types $B,C,D$.
\[
\begin{tabular}{lll}
  $G$ & $\C$ & $\BF_2$\\\hline
  $Sp(2n)$ & $\CP_{-1}(2n)$ & $\tPminus(2n)$ \\
  $O(2n+1)$ & $\CP_{1}(2n+1)$ & $\tPminus(2n)$\\
  $O(2n)$ & $\CP_{1}(2n)$ & $\tPminus(2n)$\\
\end{tabular}
\]

If $G=Sp(2n)$ then $\pi:\alpha\mapsto(\alpha,\epsilon)$ where $\epsilon(i)=1$ for all $i$ with $c_\alpha(i)>0$.

When $G=O(2n)$ we have $\pi(\alpha)=(\alpha',\epsilon)$ where $\alpha'=\inf_{\CP_{-1}(2n)}(\alpha)$
as in \cite{Spa}*{Lemma III.3.6}, or the ``C-collapse'' of $\alpha$
in the terminology of \cite{collingwood_mcgovern}.
For a description of $\epsilon$, and the case of $O(2n+1)$ see \cite{Spa}*{III.7-8}.

We will apply this in Section \ref{s:Lclassical}.


\section{Some combinatorics}

We recall a function defined in  \cite[\S 1.6]{L1}.

Suppose $\alpha=(\alpha_1,\dots,\alpha_\ell)$ is a partition. Define a function
$\psi_\alpha:\{i\mid 1\le i\le \ell\}\rightarrow\{-1,0,1\}$  as follows.
Set $\alpha_0=\alpha_{\ell+1}=0$.

\begin{enumerate}

\item If $i$ is  odd and $\alpha_{i-1}<\alpha_{i}$ then $\psi(i)=1$;
\item If $i$ is even and $\alpha_i>\alpha_{i+1}$ then $\psi(i)=-1$;
\item In all other cases $\psi(i)=0$.
\end{enumerate}

This satisfies some obvious properties. For $\ell\in\BZ$ define $\kappa_\ell\in\{0,1\}$ by: $(-1)^\ell=(-1)^{\kappa_\ell}$.

\begin{enumerate}
\item $\psi_\alpha(1)=1$; $\psi_\alpha(\ell)=-1$ if $\ell$ is even
\item If $k\le\ell$ is odd then $\sum_{i=1}^k\psi_\alpha(i)=1$
\item If $k\le\ell$ is even then $\sum_{i=1}^k\psi_\alpha(i)=1+\psi_\alpha(k)$.
\item $\sum_{i=1}^\ell\psi_\alpha(i)=\kappa_\ell$.
\end{enumerate}


Suppose $\alpha$ is a partition of $n$, and all entries $\alpha_i$ of  $\alpha$ are even. Then let
$$
\alpha+\psi_\alpha=(\alpha_1+\psi_\alpha(1),\dots, \alpha_\ell+\psi_\alpha(\ell))
$$
This is a partition of of $n+\kappa_\ell$. Explicitly,
if we write $\alpha+\psi_\alpha=(\alpha_1',\dots,\alpha_\ell')$, and
if $2i+1\le\ell$ then
$$
(\alpha'_{2i},\alpha'_{2i+1})=
\begin{cases}
(\alpha_{2i},\alpha_{2i+1})& \alpha_{2i}=\alpha_{2i+1}\\
(\alpha_{2i}-1,\alpha_{2i+1}+1)&\alpha_{2i}>\alpha_{2i+1}
\end{cases}
$$
Also $\alpha'_1=\alpha_1+1$ and, if $\ell$ is even, $\alpha'_\ell=\alpha_\ell-1$.

I think everything comes down to the following Conjecture, which is purely combinatorial.

\begin{conjecture}
  \label{c:psi}
Suppose $\beta\in\CP(n)_0$.
Set $\alpha=2\beta+\psi(\alpha)\in \CP(2n)$.
Then
\begin{enumerate}
\item $\alpha\in\CP_{1}(2n)$
\item The $\CP_{-1}$-collapse of $\alpha$ is $2\beta$.
\item $\pi(\CC_\alpha)=(2\beta,\epsilon)$ where $\epsilon(i)=1$ for all $i$.
\end{enumerate}
\end{conjecture}

Actually (1) in in \cite{L1}.

\begin{remark}
  Assuming the conjecture Lusztig's map from $[G(\C)_u]$ to $[G(\Ftwo)_u]$ agrees with
  the one defined by Spaltenstein \cite{Spa}*{Section III.7}, as asserted by Lusztig (at least
the conjugacy classes coming from $\Wec$).

\end{remark}
\end{comment}
\section{Lusztig's map for classical groups}
\label{s:Lclassical}
\subsection{The function $\psi$}
We recall a function defined in  \cite[\S 1.6]{L1}.

Suppose $\alpha=(\alpha_1,\dots,\alpha_\ell)$ is a partition. Define a function
$$\psi_\alpha:\{i\mid 1\le i\le \ell\}\rightarrow\{-1,0,1\}$$  as follows.
Set $\alpha_0=\alpha_{\ell+1}=0$.

\begin{enumerate}

\item If $i$ is  odd and $\alpha_{i-1}<\alpha_{i}$ then $\psi(i)=1$;
\item If $i$ is even and $\alpha_i>\alpha_{i+1}$ then $\psi(i)=-1$;
\item In all other cases $\psi(i)=0$.
\end{enumerate}

This satisfies some obvious properties.

For $\ell\in\BZ$ define $\kappa_\ell\in\{0,1\}$ by $(-1)^\ell=(-1)^{\kappa_\ell}$. Then

\begin{enumerate}
	\setcounter{enumi}{3}
\item $\psi_\alpha(1)=1$; $\psi_\alpha(\ell)=-1$ if $\ell$ is even;
\item If $k\le\ell$ is odd then $\sum_{i=1}^k\psi_\alpha(i)=1$;
\item If $k\le\ell$ is even then $\sum_{i=1}^k\psi_\alpha(i)=1+\psi_\alpha(k)$;
\item $\sum_{i=1}^\ell\psi_\alpha(i)=\kappa_\ell$.
\end{enumerate}


Suppose $\alpha$ is a partition of $n$, and all entries $\alpha_i$ of  $\alpha$ are even. Then let
$$
\alpha+\psi_\alpha=(\alpha_1+\psi_\alpha(1),\dots, \alpha_\ell+\psi_\alpha(\ell)).
$$
This is a partition of $n+\kappa_\ell$. Explicitly,
if we write $\alpha+\psi_\alpha=(\alpha_1',\dots,\alpha_\ell')$, and
if $2i+1\le\ell$, then
$$
(\alpha'_{2i},\alpha'_{2i+1})=
\begin{cases}
(\alpha_{2i},\alpha_{2i+1}),& \text{ if } \alpha_{2i}=\alpha_{2i+1};\\
(\alpha_{2i}-1,\alpha_{2i+1}+1),& \text{ if } \alpha_{2i}>\alpha_{2i+1}.
\end{cases}
$$
Also $\alpha'_1=\alpha_1+1$ and, if $\ell$ is even, $\alpha'_\ell=\alpha_\ell-1$.

\subsection{Parametrization of elliptic conjugacy classes}
\label{s:ellipticparam}
For classical groups the elliptic conjugacy classes in $W$ are
parametrized as follows.  Recall $\CP(n)$ is the set of partitions of
$n$,  $\CP(n)_0$ is the subset of $\CP(n)$ consisting of partitions
with an even number of parts, and $\CP(n)^{\text{odd}}$ is the set of
partitions of $n$ consisting of only odd parts.

\begin{enumerate}
	\item $G=\GLdagger(n)$. Then $G$ has two connected components: $G^0=GL(n)$ and $D=G\backslash G^0$. We have 
	\begin{itemize}
		\item $[W^{G^0}_e]$ is a singleton, and the only element is the conjugacy class of Coxeter elements;
		\item $\WDec$ is parametrized by $\CP^{odd}(n)$. 
	\end{itemize}
	\item $G=SO(2n+1)$ or $Sp(2n)$: $[W_e]$ is parametrized by $\CP(n)$. 
	\item $G=O(2n)$. Then $G$ has two connected components: $G^0=SO(2n)$ and $D=G\backslash G^0$.
          
          \begin{itemize}
\item            $[W_e]$ is parametrized by $\CP(n)$;
		\item $[W^{G^0}_e]$ is parametrized by $\CP(n)_0$; 
		\item $\WDec$ is parametrized by $\CP(n) \backslash \CP(n)_0$. 
	\end{itemize}
\end{enumerate}
For $\alpha\in\CP(n)$ we write $\CC_\alpha$ for the corresponding elliptic conjugacy class in $W$.

\subsection{Explicit description of $\Phi$ for classical groups}\label{explicit}
Suppose $G$ is a classical group. Lusztig's map $\Phi_e: \WDec \to [D_u]$ is described explicitly in
\cite[\S 4.2]{L1} and \cite[\S 3.7 \& \S 5.5]{L3}.

\begin{enumerate}
\item $G=\GLdagger(n)$. In this case, $\Phi_e$ sends the conjugacy class of Coxeter elements to the principal unipotent class. Note that both the conjugacy class of Coxeter elements and the principal unipotent class correspond to the partition $(n)$ of $n$. 

If $p=2$, then $D$ contains unipotent elements. In this case, the map $\Phi_e: \WDec \to [D_u]$ is given by $\CC_\alpha\mapsto \CU_{\alpha,\epsilonmax^\alpha}$ for $\a \in \CP^{odd}(n)$. 

\item $G=O(2n+1)$. The elliptic conjugacy classes of $W$ are parametrized by $\CP(n)$.

\begin{itemize}
\item[(i)] $p \neq 2$: the map $\Phi_e$ is given by $\CO_{\alpha} \mapsto \CU_{\alpha'}$, where
  $$
  \alpha'=\begin{cases}
2\alpha+\psi_{\alpha},    & \text{ if $\alpha$ has an odd number of parts };
\\ (2\alpha+\psi_{\alpha}, 1), & \text{ if $\alpha$ has an even number of parts}.
\end{cases}
$$
\item[(ii)] $p=2$: the map $\Phi_e$ is given by $\CO_{\alpha} \mapsto \CU_{\alpha', \epsilonmax^{\alpha'}}$
  where $\alpha'=(2\alpha,1)$.
\end{itemize}

\item $G=Sp(2n)$. The elliptic conjugacy classes of $W$ are parametrized by $\CP(n)$.
\begin{itemize}
\item[(i)]  $p \neq 2$: the map $\Phi_e$ is given by $\CO_{\alpha} \mapsto \CU_{2 \alpha}$.
\item[(ii)]  $p=2$: the map $\Phi_e$ is given by $\CO_{\alpha} \mapsto \CU_{(2 \alpha,\epsilonmax^{2\alpha})}$.
\end{itemize}

\item $G=O(2n)$. 

If $p \neq 2$, then all the unipotent elements of $G$ are contained in $G^0=SO(2n)$. In this case, $[W^{G^0}]$ is parametrized by $\CP(n)_0$. The map $\Phi_e: [W^{G^0}_e] \to [G^0_u]$ is given by $\CO_{\alpha} \mapsto \CU_{2 \alpha+\psi_{ \alpha}}$. 

If $p=2$, then $[W_e]$ is parametrized by $\CP(n)$ and the map $\Phi_e: [W_e] \to [G_u]$ is given by $\CO_{\alpha}\mapsto \CU_{2\alpha,\epsilonmax^{2\alpha}}$. 

Note that each unipotent class $\CU_{2\alpha,\epsilonmax^{2\alpha}}$ is a  single $G^0$-conjugacy classes.
\end{enumerate}

\subsection{Spaltenstein's map from characteristic $0$ to characteristic $2$}
In this subsection, let $G_p$ be a connected reductive group, defined over an
algebraically closed field $\BF$ of characteristic $p>0$. Let $G_0$
be the complex group with the same root datum as $G$.  Consider the
sets $[G_{p, u}]$ and  $[G_{0, u}]$) of unipotent conjugacy classes of $G_p$
and $G_0$, respectively.

\begin{proposition}{\cite{Spa}*{Theorem III.5.2}}
\label{p:p=2}
  There is an injective, dimension preserving map $\pi_p:[G_{0, u}]\rightarrow [G_{p, u}]$, such that $\pi_p$ is a isomorphism of partially ordered
  sets from $[G_{0, u}]$ to its image.
\end{proposition}

Here are the sets parametrizing the objects in types $B,C,D$.
\[
\begin{tabular}{lll}
  $G$ & $\C$ & $\BF_2$\\\hline
  $Sp(2n)$ & $\CP_{-1}(2n)$ & $\tPminus(2n)$ \\
  $O(2n+1)$ & $\CP_{1}(2n+1)$ & $\tPminus(2n)$\\
  $O(2n)$ & $\CP_{1}(2n)$ & $\tPminus(2n)$\\
\end{tabular}
\]


We now suppose $G$ is a classical group. It is known that $\pi_p$ is a bijection for any $p \neq 2$. Lusztig showed in \cite[Theorem 0.4 \& \S3.9]{L2} that the following diagram is commutative:
\[
\xymatrix{& [W] \ar[ld]_{\Phi_0} \ar[rd]^{\Phi_2} & \\ [G_{0, u}] \ar@{^{(}->}[rr]^{\pi_2} & & [G_{2, u}].}
\]

Thus to prove Theorem \ref{main} for $Sp(2n)$ or $SO(n)$, it suffices
to consider these groups over characteristic $2$. Note that the
explicit description of $\Phi_2$ restricted to $[W_e]$ is rather simple. We
have that

\begin{proposition}
  \label{p:Phi_elliptic}
  Let $G$ be a connected classical group defined over an algebraically
  closed field of characteristic $p \ge 0$.  Let
  $\CC_\a, \CC_\b\in\Wec$, parametrized by certain partitions as in
  Section \ref{s:ellipticparam}.  Then
  $\Phi(\CC_\a) \leu \Phi(\CC_\b)$ if and only if $\a \le \b$.
\end{proposition}

\begin{proof}
If $p \neq 2$, then $[G_{p, u}]$ and $[G_{0, u}]$ are the same as partially ordered sets.
Moreover, the map $\pi_2: [G_{0, u}] \to [G_{2,u}]$ is an
isomorphism from the poset $[G_{0, u}]$ to its image. In other
words, $\Phi_p(\CC_\a) \leu \Phi_p(\CC_\b)$ if and only if
$\Phi_2(\CC_\a) \leu \Phi_2(\CC_\b)$. Here $\Phi_p$ is the Lusztig's
map for $G$ in characteristic $p$.

	Now by \S \ref{explicit}, for $G=Sp(2n)$ or $O(n)$, the map $\Phi_2$ is given by $\CC_{\alpha} \mapsto \CU_{\alpha', \epsilonmax^{\alpha'}}$
	where $\alpha'=2 \a$ or $(2\alpha,1)$. By \S \ref{s:closure}, $\CU_{\alpha', \epsilonmax^{\alpha'}} \le \CU_{\b', \epsilonmax^{\b'}}$ if and only if $\a' \le \b'$, which is equivalent to $\a \le \b$.

A similar argument holds for $\GLdagger(n)$.
\end{proof}


An immediate consequence is that for classical groups Theorem \ref{main} is  equivalent to the following
combinatorial statement about the Weyl group, which is independent of characteristic.

\begin{proposition}
  \label{p:order_reversing}
  Suppose $W$ is a classical Weyl group.
  Suppose $\CC_\alpha,\CC_\beta\in \WDec$, for partitions $\alpha,\beta$ as in Section \ref{s:ellipticparam}. 
  Then
\begin{equation}
\label{e:order_rreversing}
\CC_\alpha\leW\CC_\beta\Leftrightarrow \beta\le\alpha.
\end{equation}
\end{proposition}

That is by Propositions \ref{p:Phi_elliptic} and \ref{p:order_reversing}
  $$
  \Phi(\CC_\alpha)\leW\Phi(\CC_\beta)\Leftrightarrow \alpha\le\beta\Leftrightarrow \CC_\beta\leW\CC_\alpha
  $$
  which proves Theorem \ref{main} for classical groups.
  


\begin{comment}
The function $\psi$ has an interpretation in terms of the map $\rho$ of the previous section (see Conjecture \ref{c:psi}).

\begin{proposition}
\label{p:psi}
Let $G=Sp(2n),O(2n+1)$ or $O(2n)$ over $\overline{\BF_2}$.
Suppose $\CC\in\Wec$ and $\CU=\Phi(\CC)\in \Gu$.

Write $\CU=\CU_{\alpha,\epsilon}$ for $(\alpha,\epsilon)\in\tPminus(2n)$.
Let $\gamma:\Gu\rightarrow \Gou$ be the map of \eqref{e:bijection}.
Then
  $$
  \gamma(\CU_{\alpha,\epsilon})=\CU_{\alpha+\psi_\alpha}\in \Gou
  $$
\end{proposition}

\begin{proof}[Sketch of proof]
  This amounts to the following. A key point is that $\CU=\Phi(\CC)$ is a strong condition
  on $\CU$. In particular, writing $\CU=\CU_{\alpha,\epsilon}$, all parts of $\alpha$ are even
  and $\epsilon(i)=1$ for all $i$ with $c_\alpha(i)\ne 0$.

  Then this requires comparing the map $\theta$ of \cite{Spa}*{III.6-8} with $\psi$.
\end{proof}

Let $\Gue\subset \Gu$ be the image of $\Phi:\Wec\rightarrow\Gu$,
and $\Goue$ similarly.  The preceding discussion can be summarized by the existence
of the commutative diagram:

$$
\xymatrix{
[W_e]\ar@{->}@/_/[r]_{\Phi_0}&\Goue\ar@/_/[l]_{\Psi_0}\ar@{->}@/_/[d]_\pi\\
[W_e]\ar@{=}[u]\ar@{->}@/_/[r]_\Phi&\Gue\ar@/_/[l]_\Psi\ar@{->}@/_/[u]_\rho
}
$$
All of the arrows in the diagram are bijections.  The bottom row is
Lusztig's map $\Phi$ for $\Gu$, restricted to the elliptic elements,
on which it has an inverse $\Psi$. The top row is the same construction
for $G_0$.  The maps $\pi,\rho$ are those of Section \ref{s:p=2},
and these are isomorphisms of partially ordered sets.

\bigskip

\begin{corollary}
\label{c:reduction} It is enough to prove Theorem \ref{t:second} in characteristic $2$.
\end{corollary}
\end{comment}


\section{Bruhat order $\leW$ for classical groups}

We give a  version of the Bruhat order for classical groups which is convenient for
our purposes.

Suppose $W\simeq S_n$ is of type $A_{n-1}$. For $w\in W$, $1\le i,j\le n$ define
$$
w[i,j]=|\{1\le k\le i\mid w(k)\ge j\}|
$$

Suppose $W\cong S_n\ltimes(\BZ/2\BZ)^n$ is of type $B_n/C_n$.
We identify $W$ with the set of permutations $\s$ of
$\{\pm 1, \pm 2, \ldots, \pm n\}$ satisfying $\s(-i)=-\s(i)$ for all
$i$. For $-n \le i, j \le n$, we define
$$
w[i, j]=|\{-n \le k \le i, w(k) \ge j\}|.
$$

Finally suppose $W\cong S_n\ltimes(\BZ/2\BZ)^{n-1}$ is of type $D_n$, and
identify $W$ with the set of permutations $\s$ of
$\{\pm 1, \pm 2, \ldots, \pm n\}$ satisfying $\s(-i)=-\s(i)$ for all
$i$, and $|\{i<-1\mid w(i)>1\}|$ is even. Define $w[i,j]$ as in type $B/C$.

\begin{proposition}
  \label{p:bruhat}
  Suppose $W$ is simple and classical, and $x,y\in W$.

\begin{enumerate}
\item   If $W$ is of type $A_{n-1}$  then $x\le y\Leftrightarrow x[i,j]\le y[i,j]$ for all $1\le i,j\le n$;
\item   If $W$ is of type $B_n/C_n$  then $x\le y\Leftrightarrow x[i,j]\le y[i,j]$ for all $-n\le i,j\le n$;
\item     If $W$ is of type $D_n$  then $x\le y\Rightarrow x[i,j]\le y[i,j]$ for all $-n\le i,j\le n$.
  \end{enumerate}

\end{proposition}

See \cite[Theorems 2.1.5, 8.1.8 and 8.2.8]{bb}.

The remainder of this section is concerned with the proof of Proposition \ref{p:order_reversing}. 



\subsection{Case 1: Type $B_n/C_n$}.



We use the following labelling of the Dynkin diagram.

$$
\begin{dynkinDiagram}[backwards,arrows=false,root radius=.12cm,edge length=-1.0cm,labels={1,2,3,n-1,n}]{B}{o3.oo}
\end{dynkinDiagram}
$$

Let $\alpha=(\alpha_1, \alpha_2, \ldots, \alpha_\ell)$ be a partition of $n$. The corresponding class $\CC_\alpha\in\Wec$ consists of the permutations
$w$ satisfying:
\begin{enumerate}
	\item There exists a decomposition $\{1, 2, \ldots, n\}=I_1 \sqcup I_2 \sqcup \ldots \sqcup I_l$;
	
	\item $|I_j|=a_j$ for all $j$;
	
	\item The orbits of $w$ on $\{\pm 1, \ldots, \pm n\}$ are $I_j \sqcup -I_j$ for $1 \le j \le l$.
\end{enumerate}

We have the following useful inequalities for the elements in $\CC_{\alpha}$.

\begin{proposition}\label{B-ineq}
  Suppose $w \in \CC_{\alpha}$ and $0 \le m \le n$. Then
$$
|w[n-m, n-m+1]| \ge \min\{k; a_1+\cdots+a_k \ge m\}.
$$
\end{proposition}

\begin{proof}
Let $I_1, \ldots, I_l$ be the subsets of $\{1, \ldots, n\}$
satisfying the conditions (1)-(3) for $w$. Let $1 \le j \le l$ and
$r=\max\{i\mid i\in I_j\}$.  Then $w^{|I_j|}(-r)=r$. Therefore
if $r \ge n-m+1$, then there exists $s \in \BN$ such that
$w^s(-r) \le n-m$ and $w^{s+1}(-r) \ge n-m+1$. In other
words,
$$
w[n-m, n-m+1] \cap (I_j \sqcup -I_j) \neq \emptyset.
$$
Therefore
$$
|w[n-m, n-m+1]| \ge |\{j; \max I_j \ge n-m+1\}|.
$$


Let $J=\{j; \max I_j \ge n-m+1\}$. Note that there are exactly $m$ elements in $I_1 \sqcup \ldots \sqcup I_l$ that are larger than or equal to $n-m+1$. We have $\sum_{j \in J} | I_j| \ge m$. As $|I_k|=a_k$ for all $k$ and $a_1 \ge \cdots \ge a_l$, we have $a_1+\cdots+a_{| J|} \ge \sum_{j \in J} | I_j| \ge m$. In other words, $| J| \ge \min\{k; a_1+\cdots+a_k \ge m\}$. The proposition is proved.
\end{proof}

Following \cite{He07}\footnote{Note that the convention for the labelling of the simple reflections here is opposite to the labelling used in \cite{He07} and the formulas belows are modified accordingly.}, for $1 \le a, b \le n$, we define $$s_{[a, b]}=\begin{cases} s_a s_{a+1} \cdots s_b, & \text{ if } a \le b; \\ 1, & \text{ otherwise}.\end{cases}$$

For any partition $\alpha=(a_1, \ldots, a_l)$ of $n$, we define $$w_{\alpha}=(s_{[2, n+1-a_1]} \i s_{[1, n]}) (s_{[2, n+1-a_1-a_2]} \i s_{[1, n-a_1]}) \cdots (s_{[1, a_l]}).$$ By \cite[\S 7]{He07}, $w_{\alpha}$ is a minimal length element in $\CC_{\alpha}$.

\begin{proof}[Proof of Proposition \ref{p:order_reversing} for type B/C]



If $\alpha \ge \beta$, then $\alpha_1 \ge \beta_1$, $\alpha_1+\alpha_2 \ge \beta_1+\beta_2$, $\ldots$. Thus \begin{gather*} s_{[2, n+1-\alpha_1]} \i s_{[1, n]} \le s_{[2, n+1-\beta_1]} \i s_{[1, n]}, \\ s_{[2, n+1-\alpha_1-\alpha_2]} \i s_{[1, n-\alpha_1]} \le s_{[2, n+1-\beta_1-\beta_2]} \i s_{[1, n-\beta_1]}, \\ \ldots. \end{gather*} So $w_{\alpha} \le w_{\beta}$ and thus $O_{\alpha} \leW O_{\beta}$.
	
On the other hand, if $O_{\alpha} \leW O_{\beta}$, then
there exists $w \in O_{\alpha}$ with $w \le w_{\beta}$. Let
$m=\beta_1+\cdots+\beta_k$. By direct computation, we have
$\sharp w_{\beta}[n-m, n-m+1]=k$. By Proposition
\ref{p:bruhat}, $\sharp w[n-m, n-m+1] \le k$. By Proposition
\ref{B-ineq}, $k \ge \min\{k'; \alpha_1+\cdots+\alpha_{k'} \ge m\}$. In other
words, $\alpha_1+\cdots+\alpha_k \ge m=\beta_1+\cdots+\beta_k$. Therefore
$\alpha \ge \beta$.
\end{proof}

\subsection{Case 2: Type $D_n$}
Let $W$ be the Weyl group of type $D_n$. We identify $W$ with the permutations $w$ of $\{\pm1, \dots, \pm n\}$ such that $w(-j) = -w(j)$ and $\sharp w[-1, 1]$ is even.

Let $\d$ be the permutation of $\{\pm1, \dots, \pm n\}$ such that $\d(1) = -1$ and $\d(k) = k$ for $2 \le k \le n$. The map $w \mapsto \d(w) := \d w \d\i$ is an outer involution of $W$. The map $W \to W\d: w \mapsto w\d$ induces a bijection from the set of $\d$-twisted conjugacy class of $W$ to the set of ordinary $W$-conjugacy classes of $W\d$.
\begin{proposition} \label{D-i-j-d}
Let $x, y \in W$. If $x \leq y$, then $\sharp \d x[i, j] \le \sharp \d y[i, j]$ for any $-n \le i, j \le n$.
\end{proposition}
\begin{proof}
Suppose otherwise. Then $\sharp \d x[a, b] > \sharp \d y[a, b]$ for some $-n \le a, b \le n$. For $z \in W$ one checks that $-1 \le \sharp z[i, j] - \sharp \d z[i, j] \le 1$, and moreover, $z[i, j] = \d z[i, j]$ (resp. $z[i, j] = z \d[i, j]$) if $j \neq 1$ (resp. $i \neq -1$).  As $x \leq y$ and $\d(x) \leq \d(y)$, we have $\sharp x[a, b] \le \sharp y[a, b]$ and $\sharp \d(x)[a, b] \le \sharp \d(y)[a, b]$ by Proposition \ref{D-i-j}. So we deduce that $a = -1$, $b = 1$ and $0 \le \sharp y[-1, 1] - \sharp x[-1, 1] \le 1$. Noticing that both $\sharp x[-1, 1]$ and $\sharp y[-1, 1]$ are even, we have $\sharp x[-1, 1] = \sharp y[-1, 1]$. Let $c = \min \{x(k), -1 \le k \le n, x(k) \ge 1\}$ and $d = \min \{y(k), -1 \le k \le n, y(k) \ge 1\}$. In particular, $d \ge c \ge 1$ as $\sharp x[-1, 1] = \sharp y[-1, 1]$.

If $d > 1$, then $\sharp y[-1, 1] = \sharp \d y[-1, 1] = \sharp \d x[-1, 1] - 1$, which means $c > 1$ and $x\i(-1) \le -1 < y\i(-1)$. If $d = 1$, then $c = 1$ and hence $\sharp \d x[-1, 1] = \sharp x[-1, 1] = \sharp \d y[-1, 1] + 1$, which also means $x\i(-1) \le -1 < y\i(-1)$. In either case, we have $\sharp x[-1, -1] = \sharp x[-1, 1] + 1 > \sharp y[-1, 1] = \sharp y[-1, -1]$, a contradiction.
\end{proof}


The elliptic conjugacy classes (resp. $\d$-twisted conjugacy classes) of $W$ are parameterized by the partitions of $n$ of even (resp. odd) lengths. Let $\alpha=(\alpha_1, \alpha_2, \ldots, \alpha_l)$ be a partition of $n$ of even length, i.e. $\alpha_1 \ge \alpha_2 \ge \cdots \ge \alpha_l>0$, $\alpha_1+\cdots+\alpha_l=n$ and $l$ is even (resp. odd). Let $O_{\alpha}$ (resp. ${}^2 O_{\alpha}$) be the corresponding elliptic conjugacy class (resp. $\d$-twisted conjugacy classes) of $W$. This is the conjugacy class (resp. $\d$-twisted conjugacy classe) of $W$  consisting of permutations $w$ satisfying the following conditions:

\begin{enumerate}
	\item There exists a decomposition $\{1, 2, \ldots, n\}=I_1 \sqcup I_2 \sqcup \ldots \sqcup I_l$;
	
	\item $\sharp I_j=\alpha_j$ for all $j$;
	
	\item The orbits of $w$ (resp. $w\d$) on $\{\pm 1, \ldots, \pm n\}$ are $I_j \sqcup -I_j$ for $1 \le j \le l$.
\end{enumerate}

\begin{proposition}\label{D-ineq}
For $0 \le m \le n$, we have \begin{align*}\sharp w[n-m, n-m+1] & \ge \min\{k; \alpha_1+\cdots+\alpha_k \ge m\} \text{ if } w \in O_{\alpha}; \\ \sharp w\d[n-m, n-m+1] & \ge \min\{k; \alpha_1+\cdots+\alpha_k \ge m\} \text{ if } w \in {}^2 O_{\alpha}. \end{align*}
\end{proposition}
\begin{proof}
It follows similarly as Proposition \ref{B-ineq}.
\end{proof}


\begin{proposition}\label{D-main}
Let $\alpha, \beta$ be even (resp. odd) partitions of $n$. Then $O_{\alpha} \leW O_{\beta}$ (resp. ${}^2 O_{\alpha} \leW {}^2 O_{\beta}$) if and only if $\alpha \ge \beta$.
\end{proposition}

\begin{proof}
  We only consider the odd partition case. If
  $\alpha \ge \beta$, then
  ${}^2 O_{\alpha} \leW {}^2 O_{\beta}$ follows similarly
  as in Proposition \ref{B-main} from the explicit constructions of
  minimal length elements $w_{\alpha}$ and $w_{\beta}$ in
  ${}^2 O_{\alpha}$ and ${}^2 O_{\alpha}$ respectively.
	
On the other hand, if ${}^2 O_{\alpha}\d \leW {}^2 O_{\beta}$, then there exists $w \in {}^2 O_{\alpha}$ with $w \leq w_{\beta}$. Let $m=\beta_1+\cdots+\beta_k$. By direct computation, we have $\sharp w_{\beta} \d[n-m, n-m+1]=k$. By Proposition \ref{D-i-j-d}, $\sharp w\d[n-m, n-m+1] \le k$. By Proposition \ref{D-ineq}, $k \ge \min\{k'; \alpha_1+\cdots+\alpha_{k'} \ge m\}$. In other words, $\alpha_1+\cdots+\alpha_k \ge m=\beta_1+\cdots+\beta_k$. Therefore $\alpha \ge \beta$.
\end{proof}

\subsection{Type ${}^2 A_{n-1}$} Let $W=S_n$ be the group of permutations of $\{1, 2, \cdots, n\}$. For $1 \le i, j \le n$, we define $$w[i, j]=\{-n \le k \le n; k \le i, w(k) \ge j\}.$$ We have the following explicit description of the Bruhat order. \remind{reference}

\begin{proposition}\label{A-i-j}
Let $x, y \in W$. Then $x \le y$ if and only if $\sharp x[i, j] \le \sharp y[i, j]$ for any $1 \le i, j \le n$.
\end{proposition}

Let $\d=(1, n) (2, n-1) \cdots$ be the longest element of $W$. Then the conjugation action of $\d$ on $W$ induces a bijection on the set of simple reflections and is a length-preserving automorphism. The map $W \to W: w \mapsto w \d$ induces a bijection from the set of $\d$-twisted conjugacy class of $W$ to the set of ordinary conjugacy classes of $W$. Since $\d$ is the longest element of $W$, the map is order-reversing.

The $\d$-twisted elliptic conjugacy classes of $W$ are parametrized by the partitions of $n$ with odd parts. Let $\alpha=(\alpha_1, \alpha_2, \ldots, \alpha_l)$ be a partition of $n$ with odd parts, i.e. $\alpha_1 \ge \alpha_2 \ge \cdots \ge \alpha_l>0$ are odd positive integers and $\alpha_1+\cdots+\alpha_l=n$. Let $O_{\alpha}$ be the corresponding $\d$-twisted elliptic conjugacy class in $W$. This is the conjugacy class of $W$ consisting of permutations $w$ satisfying the following conditions:

\begin{enumerate}
	\item There exists a decomposition $\{1, 2, \ldots, n\}=I_1 \sqcup I_2 \sqcup \ldots \sqcup I_l$;
	
	\item $\sharp I_j=\alpha_j$ for all $j$;
	
	\item The orbits of $w \d$ on $\{1, 2, \ldots, n\}$ are $I_j$ for $1 \le j \le l$.
\end{enumerate}

We have the following useful inequalities for the elements in $O_{\alpha}$.

\begin{proposition}\label{A-ineq}
	Let $w \in O_{\alpha}$ and $1 \le m \le n-1$, we have $$\sharp w\d[\lceil \frac{m}{2}\rceil, n-\lfloor \frac{m}{2} \rfloor+1]+\sharp w\d[n-\lfloor \frac{m}{2} \rfloor, \lceil \frac{m}{2}\rceil+1] \le n-\min\{k; \alpha_1+\cdots+\alpha_k \ge m\}.$$
\end{proposition}

\begin{proof} Let $I=w\d[\lceil \frac{m}{2}\rceil, n-\lfloor \frac{m}{2} \rfloor+1]$ and $I'=w\d[n-\lfloor \frac{m}{2} \rfloor, \lceil \frac{m}{2}\rceil+1]$. If $k \in I$, then $k \le \lceil \frac{m}{2}\rceil<\lceil \frac{m}{2}\rceil+1$ and thus $(w \d)\i (k) \notin I'$. In other words, $$I \cap (w \d) (I')=\emptyset.$$ Similarly,  $$I \cap (w \d) \i (I')=\emptyset.$$
	
In particular, for any $w\d$-orbit $I_j$, we have \[\tag{a}\sharp (I_j \cap I)+\sharp (I_j \cap I')=\sharp (I_j \cap I)+\sharp (I_j \cap (w \d)(I')) \le \sharp (I_j).\]

We claim that

(b) If $\sharp (I_j \cap I)+\sharp (I_j \cap I')= \sharp(I_j)$, then $I_j \subset \{\lceil \frac{m}{2}\rceil+1, \lceil \frac{m}{2}\rceil+2, \ldots, n-\lfloor \frac{m}{2} \rfloor\}$.

Note that $$\sharp(I_j \cap I)+\sharp(I_j \cap w \d(I'))=\sharp(I_j \cap I)+\sharp(I_j \cap (w \d) \i (I'))=\sharp(I_j).$$ Since $I \cap (w \d)(I')=I \cap (w \d) \i(I')=\emptyset$, we have $$(I_j \cap I) \sqcup (I_j \cap (w \d)(I'))=(I_j \cap I) \sqcup (I_j \cap (w \d) \i(I'))=I_j.$$ Thus $$(w \d) (I_j \cap I')=I_j \cap (w \d)(I')=I_j \cap (w \d) \i(I')=(w \d) \i(I_j \cap I').$$ In other words, $I_j \cap I'$ is a subset of $I_j$ that is stable under the action of $(w \d)^2$. Since the order of the action of $w \d$ on $I_j$ equals to $\sharp I_j$, which is an odd integer. Hence $I_j \cap I'$ is a $w \d$-stable subset of $I_j$. As $w \d$ acts transitively on $I_j$, we have $I_j \cap I'=\emptyset$ or $I_j$. Hence $I_j \subset I$ or $I_j \subset I'$. However, as $\lceil \frac{m}{2} \rceil<n-\lfloor \frac{m}{2} \rfloor+1$, if $k \in I$, then $(w \d)(k) \notin I$. Thus $I_j \subset I'$. In other words, for any $k \in I_j$, $k \ge \lceil \frac{m}{2} \rceil+1$ and $k \le n-\lfloor \frac{m}{2} \rfloor$.

(b) is proved.

Let $J=\{j; I_j \not\subset \{\lceil \frac{m}{2}\rceil+1, \lceil \frac{m}{2}\rceil+2, \ldots, n-\lfloor \frac{m}{2} \rfloor\}\}$. By (a) and (b), we have that $$\sharp J+\sharp J'=\sum_j (\sharp (I_j \cap J)+\sharp(I_j \cap J')) \le \sum_j \sharp(I_j)-\sharp J=n-\sharp J.$$ Note that there are exactly $m$ elements outside $\{\lceil \frac{m}{2}\rceil+1, \lceil \frac{m}{2}\rceil+2, \ldots, n-\lfloor \frac{m}{2} \rfloor\}$. We have $\sum_{j \in J} \sharp I_j \ge m$. As $\sharp I_k=\alpha_k$ for all $k$ and $\alpha_1 \ge \cdots \ge \alpha_l$, we have $\alpha_1+\cdots+\alpha_{\sharp J} \ge \sum_{j \in J} \sharp I_j \ge m$. In other words, $\sharp J \ge \min\{k; \alpha_1+\cdots+\alpha_k \ge m\}$. The proposition is proved.
\end{proof}

Now we prove the following result.

\begin{proposition}
	Let $\alpha, \beta$ be partitions of $n$ with odd parts. Then $O_{\alpha} \leW O_{\beta}$ if and only if $\alpha \ge \beta$.
\end{proposition}

\begin{proof}
	The strategy is similar to the proof of Proposition \ref{B-main}.
	
	We use the explicit minimal length elements $w_{\alpha}$ of $\CC_{\alpha}$ constructed in \cite[Lemma 7.13]{He07}.
	If $\alpha \ge \beta$, then $\alpha_1 \ge \beta_1$, $\alpha_1+\alpha_2 \ge \beta_1+\beta_2$, $\ldots$. By the explicit formula, $w_{\alpha} \le w_{\beta}$ and thus $\CC_{\alpha} \leW \CC_{\beta}$.
	
	On the other hand, if $\CC_{\alpha} \leW \CC_{\beta}$, then there exists $w \in \CC_{\alpha}$ with $w \le w_{\beta}$. Hence $w \d \ge w_{\beta} \d$. Let $m=\beta_1+\cdots+\beta_k$. By direct computation, we have $$\sharp w_{\beta} \d[\lceil \frac{m}{2}\rceil, n-\lfloor \frac{m}{2} \rfloor+1]+\sharp w_{\beta} \d[n-\lfloor \frac{m}{2} \rfloor, \lceil \frac{m}{2}\rceil+1]=n-k.$$
	
	By Proposition \ref{A-i-j}, $$\sharp w \d[\lceil \frac{m}{2}\rceil, n-\lfloor \frac{m}{2} \rfloor+1]+\sharp w \d[n-\lfloor \frac{m}{2} \rfloor, \lceil \frac{m}{2}\rceil+1]\ge n-k.$$ By Proposition \ref{A-ineq}, $k \ge \min\{k'; \alpha_1+\cdots+\alpha_{k'} \ge m\}$. In other words, $\alpha_1+\cdots+\alpha_k \ge m=\beta_1+\cdots+\beta_k$. Therefore $\alpha \ge \beta$.
\end{proof}

\section{Examples}

\subsection{$\Sp(4)$}

\bigskip\hfil


\begin{tabular}{lll}
  $[W]$ & $Sp(4,\C)$ & $\Sp(4,\overline\BF_2)$\\
  $[2]$ & $[4]$& $([4],*)$ \\
  $[1,1]$ & $[2,2]$& $([2,2],\epsilon(2)=1)$ \\
  \hline
  $(A^s_1,*)$ & $[2,2]$ &$([2,2], \epsilon(2)=0)$ \\
  $(A^l_1,*)$ & $[2,1,1]$& $([2,1,1], *)$\\
  $(T,*)$ & $[1,1,1,1]$ & $([1,1,1,1],*)$
\end{tabular}
\begin{comment}

\bigskip

$\O(4)$

The unipotent orbits in $\O(4,\overline \BF_2)$ are parametrized
by $\tPminus(4)$.

The Weyl group of $\O(4)$ is of type $C_2$ which has five conjugacy
classes, two of which are elliptic. We borrow the parametrization of
these conjugacy classes from $W(C_2)$.

\begin{tabular}{llll}
  $[W]$ & $O(4,\C)$ & $\O(4,\overline\BF_2)$& $\SO/\O$\\
    \hline\hline
  $[2]$ & $[3,1]$& $([4], *)$ & $\O$\\
  $[1,1]$ & $[2,2]$& $([2,2], \epsilon(2)=1)$ &$\SO$\\
  \hline
   $(A^s_1,*)$ & $[2,2]$ &$([2,2], \epsilon(2)=0)$&$\SO$\\
  $(A^l_1,*)$ & $[2,1,1]$& $([2,1,1],*)$&$\O$\\
  $(T,*)$ & $[1,1,1,1]$ & $([1,1,1,1],*)$ & $\SO$
\end{tabular}

\bigskip
\end{comment}

\subsection{$\SO(4)$}

\hfil

The unipotent orbits in $\SO(4,\overline \BF_2)$ are parametrized
by $\tPminus(4)_0$ (odd rows have even multiplicity, the first column has even length).
Also a strongly even partitions count twice provided $\epsilon(i)=0$ for all $i$;
these are denote with a subscript $I$ or $II$.

There is one elliptic class in $W$, $4$ total. Also  $\SO(4,\C)$  and $\SO(4,\Ftwo)$ each have $4$ unipotent orbits.

\begin{tabular}{lll}
  $[W]$ & $SO(4,\C)$ & $\SO(4,\overline\BF_2)$\\
  \hline\hline
  $[1,1]$ & $[3,1]$& $([2,2], \epsilon(2)=1)$ \\
    \hline
  $(A_1,*)$ & $[2,2]_I$ &$([2,2], \epsilon(2)=0)_I$ \\

  $(A'_1,*)$ & $[2,2]_{II}$& $([2,2], \epsilon(2)=0)_{II}$ \\
  $(T,*)$ & $[1,1,1,1]$ & $([1,1,1,1],*)$
\end{tabular}

\newpage

\subsection{$\SO(6)$}

\hfil

The Weyl group of $\SO(6)$ is of type $D_3$ which has $5$ conjugacy
classes, $1$ of which is elliptic.

\begin{tabular}{lll}
  $[W]$ & $SO(6,\C)$ & $\SO(6,\overline\BF_2)$\\
    \hline\hline
  $[2,1]$ & $[5,1]$& $([4,2], *)$\\
  \hline
  $(A_2,*)$ & $[3,3]$ & $([3,3],*)$\\
  $(D_2,*)$ & $[3,1,1,1]$& $([2,2,1,1],\epsilon(2)=1)$\\
  $(A_1,*)$ &  $[2,2,1,1]$& $([2,2,1,1],\epsilon(2)=0)$\\
  $(T,*)$ &  $ [1,1,1,1,1,1]$& $([1,1,1,1,1,1],*)$\\
\end{tabular}

\bigskip

\subsection{$\SO(8)$}

\hfil

The Weyl group of $\SO(8)$ is of type $D_3$, which has $13$ conjugacy
classes, $3$ of which are elliptic.
Both $\SO(8,\C)$ and $\SO(8,\Ftwo)$ have $12$ unipotent classes.
Note that the orbits $[3,2,2,1]$ and $([2,2,2,2],\epsilon(2)=1)$ both occur twice in the image.


Note that $SO(8)$ has two non-conjugate $GL(4)=A_3$ Levi factors
in addition to $GL(1)\times SO(6)$ of type $D_3$.
Also it has
two non-conjugate $\GL(2)\times \GL(2)$ factors.

Also a unipotent orbit for $\O(8,\Ftwo)$ splits into two
for $\SO(8,\Ftwo)$ if and only if all parts are even, with even multiplicity, and all $\epsilon(i)=0$.


\begin{tabular}{llll}
$\#$ & $[W]$ & $SO(8,\C)$ & $\SO(8,\overline\BF_2)$\\
1&  $[3,1]$ & $[7,1]$& $([6,2], *)$\\
2&  $[2,2]$ & $[5,3]$& $([4,4], \epsilon(4)=1)$\\
  3&  $[1,1,1,1]$ & $[3,2,2,1]$& $([2,2,2,2], \epsilon(2)=1)$\\
  \hline
4&  $D_3=SO(6)\times GL(1)$ & $[5,1,1,1]$& $[4,2,1,1]$\\
5&  $A_3=GL(4)$ & $[4,4]_I$ & $([4,4],\epsilon(4)=0)_I$\\
6&  $A'_3=GL(4)'$ & $[4,4]_{II}$ &$([4,4],\epsilon(4)=0)_{II}$\\
7&  $A_1\times D_2=GL(2)\times SO(4)$ & $[3,2,2,1]$ & $([2,2,2,2],\epsilon(2)=1)$\\
8&  $D_2=\GL(1)\times \SO(4)$ & $[3,1,1,1,1,1]$ & $([2,2,1,1,1,1],\epsilon(2)=1)$\\
9&  $A_2=\GL(3)\times \GL(1)$&$[3,3,1,1]$ & $([3,3,1,1],*)$\\
10&  $2A_1=\GL(2)\times \GL(2)$&$[2,2,2,2]_I$ & $([2,2,2,2],\epsilon(2)=0)_I$\\
11&    $2A_1=\GL(2)\times \GL(2)'$&$[2,2,2,2]_{II}$& $([2,2,2,2],\epsilon(2)=0)_{II}$\\
12&  $A_1=\GL(2)\times\GL(1)\times\GL(1)$&$[2,2,1,1,1,1]$ & $([2,2,1,1,1,1,1,1],\epsilon(2)=0)$\\
13&$T$ & $[1^8]$ & $([1^8],*)$
\end{tabular}

\newpage

\subsection{$\SO(12)$}

\hfil

Let's just look at the elliptic classes.

\begin{tabular}{llll}
  $\#$ & $[W]$ & $SO(12,\C)$ & $\SO(12,\overline\BF_2)$\\
  1& $[5,1]$ & $[11,1]$ & $([10,2],*)$\\
  2& $[4,2]$ & $[9,3]$ & $([8,4],*)$\\
  3& $[3,3]$ & $[7,5]$ & $([6,6],\epsilon(6)=1))$\\
  4& $[2,2,1,1]$ & $[5,3,3,1]$ & $([4,4,2,2],\epsilon(4)=\epsilon(2)=1)$\\
    5& $[1,1,1,1,1,1]$ & $[3,2,2,2,2,1]$ & $([2,2,2,2,2,2],\epsilon(2)=1)$\\
\end{tabular}

\begin{thebibliography}{XXXX99}

\bibitem[BB05]{bb}
  F. Brenti and A. Bj\"{o}rner, \emph{Combinatorics of Coxeter groups},
  Graduate Texts in Mathematics, vol. 231, Springer, New York (2005)

\bibitem[CM93]{collingwood_mcgovern}
  D. Collingwood and M. McGovern, \emph{Nilpotent orbits in semisimple Lie algebras},
  Van Nostrand Reinhold Mathematics Series, Van Nostrand Reinhold Co., New York (1993)

\bibitem[DHM13]{DHM}
O. Dudas, X. He and J. Michel, \emph{Private communication}, 2013.

\bibitem[DL76]{DL}
P. Deligne and G. Lusztig, \emph{Representations of reductive groups over finite fields}, Ann. of Math. (2) 103 (1976), no. 1, 103--161.

\bibitem[DM15]{DM}
O. Dudas and G. Malle, \emph{Decomposition matrices for low-rank unitary groups}, Proc. Lond. Math. Soc. (3) 110 (2015), no. 6, 1517--1557.

\bibitem[JO00]{JO}
A. J. de Jong and F. Oort, \emph{Purity of the stratification by Newton polygons}, J. Amer. Math. Soc. 13 (2000), 209--241.

\bibitem[Lu11]{L1}
G. Lusztig, \emph{From conjugacy classes in the Weyl group to unipotent classes}, Represent. Theory 15 (2011), 494--530.

\bibitem[Lu12a]{L2}
G. Lusztig, \emph{From conjugacy classes in the Weyl group to unipotent classes, II}, Represent. Theory 16 (2012), 189--211.

\bibitem[Lu12b]{L3}
G. Lusztig, \emph{From conjugacy classes in the Weyl group to unipotent classes, III}, Represent. Theory 16 (2012), 450--488.

\bibitem[Ha15]{Ha}
P. Hamacher, \emph{The geometry of Newton strata in the reduction modulo p of Shimura varieties of PEL type}, Duke Math. J. 164 (2015), no. 15, 2809--2895.

\bibitem[HV11]{HV}
U. Hartl and E. Viehmann, \emph{The Newton stratification on deformations of local G-shtukas}, J. Reine Angew. Math. 656 (2011), 87--129.

\bibitem[He07]{He07}
X. He, {\em Minimal length elements in some double cosets of Coxeter groups},  Adv. Math. 215 (2007), 469--503.

\bibitem[He14]{He-Ann}
X. He, \emph{Geometric and homological properties of affine Deligne-Lusztig varieties}, Ann. of Math. (2) 179 (2014), 367--404.

\bibitem[He16a]{He-CDM}
X. He, \emph{Hecke algebras and $p$-adic groups}, Current developments in mathematics 2015, 73--135, Int. Press, Somerville, MA, 2016.

\bibitem[He16b]{He-KR}
X. He, \emph{Kottwitz-Rapoport conjecture on unions of affine Deligne-Lusztig varieties},  Ann. Sci. \`Ecole Norm. Sup. 49 (2016), 1125--1141.

\bibitem[HN14]{HN2}
X. He and S. Nie, \emph{Minimal length elements of extended affine Weyl group}, Compos. Math. 150 (2014), 1903--1927.

\bibitem[Sp82]{Spa}
N. Spaltenstein, \emph{Classes unipotents et sous-groupes de Borel}, Lecture Notes in Math. 946, Springer, 1982.

\bibitem[Vi13]{Vi}
E. Viehmann, \emph{Newton strata in the loop group of a reductive group}, Amer. J. Math. 135 (2013), no. 2, 499--518.

\bibitem[Wa63]{Wall}
G. E. Wall, \emph{On the conjugacy classes in the unitary, symplectic and orthogonal groups},  J. Austral. Math. Soc., 3 (1963), 1--63.

\end{thebibliography}


\end{document}
