\documentclass[a4paper, reqno, 12pt]{amsart}
% alternative choice:
%\documentclass[a4paper, reqno, 12pt]{article}
%\usepackage{authblk}    %comment this out when using amsart


\usepackage{amsthm,amsfonts,amssymb,amsmath,amsxtra,amsrefs,array,dynkin-diagrams}
\usepackage{xr-hyper}
\usepackage[matrix,tips,frame,color,line,poly]{xy}
%\usepackage[colorlinks=
%   citecolor=Black,
%   linkcolor=Red,
%   urlcolor=Blue]{hyperref}
\usepackage{verbatim}
\usepackage{tikz}
\usepackage[margin=1.25in]{geometry}
\usepackage{mathrsfs}
%\usepackage{showkeys}
\setlength\extrarowheight{6pt}  %extra height in tables, using array package
\newcommand{\remind}[1]{{\bf ** #1 **}}

\RequirePackage{xspace}
% load etoolbox package, for programming features
\RequirePackage{etoolbox}
% load varwidth package, for text environments which are automatically the natural width of the text they contain
\RequirePackage{varwidth}
% load enumitem package, for easy margin adjustment in enumerate and itemize environments
\RequirePackage{enumitem}
% load tensor package, for good placement of super/subscripts to the left of symbols
\RequirePackage{tensor}
% load mathtools package, for various extensions of amsmath
\RequirePackage{mathtools}
% load longtable package, which allows tables to (if needed) split over multiple pages
\RequirePackage{longtable}
% load multirow package, which allows cells spanning multiple rows in tables
\RequirePackage{multirow}



% put sections only (as opposed to subsections) in the table of contents
\setcounter{tocdepth}{1}


\def\ge{\geqslant}
\def\le{\leqslant}
\def\a{\alpha}
\def\b{\beta}
\def\g{\gamma}
\def\G{\Gamma}
\def\d{\delta}
\def\D{\Delta}
\def\L{\Lambda}
\def\e{\epsilon}
\def\et{\eta}
\def\io{\iota}
\def\o{\omega}
\def\p{\pi}
\def\ph{\phi}
\def\ps{\psi}
%\def\r{\rho}
\def\s{\sigma}
\def\t{\tau}
\def\th{\theta}
\def\k{\kappa}
\def\l{\lambda}
\def\z{\zeta}
\def\v{\vartheta}
\def\x{\xi}
\def\i{^{-1}}

\def\<{\langle}
\def\>{\rangle}

\newcommand{\sA}{\ensuremath{\mathscr{A}}\xspace}
\newcommand{\sB}{\ensuremath{\mathscr{B}}\xspace}
\newcommand{\sC}{\ensuremath{\mathscr{C}}\xspace}
\newcommand{\sD}{\ensuremath{\mathscr{D}}\xspace}
\newcommand{\sE}{\ensuremath{\mathscr{E}}\xspace}
\newcommand{\sF}{\ensuremath{\mathscr{F}}\xspace}
\newcommand{\sG}{\ensuremath{\mathscr{G}}\xspace}
\newcommand{\sH}{\ensuremath{\mathscr{H}}\xspace}
\newcommand{\sI}{\ensuremath{\mathscr{I}}\xspace}
\newcommand{\sJ}{\ensuremath{\mathscr{J}}\xspace}
\newcommand{\sK}{\ensuremath{\mathscr{K}}\xspace}
\newcommand{\sL}{\ensuremath{\mathscr{L}}\xspace}
\newcommand{\sM}{\ensuremath{\mathscr{M}}\xspace}
\newcommand{\sN}{\ensuremath{\mathscr{N}}\xspace}
\newcommand{\sO}{\ensuremath{\mathscr{O}}\xspace}
\newcommand{\sP}{\ensuremath{\mathscr{P}}\xspace}
\newcommand{\sQ}{\ensuremath{\mathscr{Q}}\xspace}
\newcommand{\sR}{\ensuremath{\mathscr{R}}\xspace}
\newcommand{\sS}{\ensuremath{\mathscr{S}}\xspace}
\newcommand{\sT}{\ensuremath{\mathscr{T}}\xspace}
\newcommand{\sU}{\ensuremath{\mathscr{U}}\xspace}
\newcommand{\sV}{\ensuremath{\mathscr{V}}\xspace}
\newcommand{\sW}{\ensuremath{\mathscr{W}}\xspace}
\newcommand{\sX}{\ensuremath{\mathscr{X}}\xspace}
\newcommand{\sY}{\ensuremath{\mathscr{Y}}\xspace}
\newcommand{\sZ}{\ensuremath{\mathscr{Z}}\xspace}


\newcommand{\fka}{\ensuremath{\mathfrak{a}}\xspace}
\newcommand{\fkb}{\ensuremath{\mathfrak{b}}\xspace}
\newcommand{\fkc}{\ensuremath{\mathfrak{c}}\xspace}
\newcommand{\fkd}{\ensuremath{\mathfrak{d}}\xspace}
\newcommand{\fke}{\ensuremath{\mathfrak{e}}\xspace}
\newcommand{\fkf}{\ensuremath{\mathfrak{f}}\xspace}
\newcommand{\fkg}{\ensuremath{\mathfrak{g}}\xspace}
\newcommand{\fkh}{\ensuremath{\mathfrak{h}}\xspace}
\newcommand{\fki}{\ensuremath{\mathfrak{i}}\xspace}
\newcommand{\fkj}{\ensuremath{\mathfrak{j}}\xspace}
\newcommand{\fkk}{\ensuremath{\mathfrak{k}}\xspace}
\newcommand{\fkl}{\ensuremath{\mathfrak{l}}\xspace}
\newcommand{\fkm}{\ensuremath{\mathfrak{m}}\xspace}
\newcommand{\fkn}{\ensuremath{\mathfrak{n}}\xspace}
\newcommand{\fko}{\ensuremath{\mathfrak{o}}\xspace}
\newcommand{\fkp}{\ensuremath{\mathfrak{p}}\xspace}
\newcommand{\fkq}{\ensuremath{\mathfrak{q}}\xspace}
\newcommand{\fkr}{\ensuremath{\mathfrak{r}}\xspace}
\newcommand{\fks}{\ensuremath{\mathfrak{s}}\xspace}
\newcommand{\fkt}{\ensuremath{\mathfrak{t}}\xspace}
\newcommand{\fku}{\ensuremath{\mathfrak{u}}\xspace}
\newcommand{\fkv}{\ensuremath{\mathfrak{v}}\xspace}
\newcommand{\fkw}{\ensuremath{\mathfrak{w}}\xspace}
\newcommand{\fkx}{\ensuremath{\mathfrak{x}}\xspace}
\newcommand{\fky}{\ensuremath{\mathfrak{y}}\xspace}
\newcommand{\fkz}{\ensuremath{\mathfrak{z}}\xspace}


\newcommand{\fkA}{\ensuremath{\mathfrak{A}}\xspace}
\newcommand{\fkB}{\ensuremath{\mathfrak{B}}\xspace}
\newcommand{\fkC}{\ensuremath{\mathfrak{C}}\xspace}
\newcommand{\fkD}{\ensuremath{\mathfrak{D}}\xspace}
\newcommand{\fkE}{\ensuremath{\mathfrak{E}}\xspace}
\newcommand{\fkF}{\ensuremath{\mathfrak{F}}\xspace}
\newcommand{\fkG}{\ensuremath{\mathfrak{G}}\xspace}
\newcommand{\fkH}{\ensuremath{\mathfrak{H}}\xspace}
\newcommand{\fkI}{\ensuremath{\mathfrak{I}}\xspace}
\newcommand{\fkJ}{\ensuremath{\mathfrak{J}}\xspace}
\newcommand{\fkK}{\ensuremath{\mathfrak{K}}\xspace}
\newcommand{\fkL}{\ensuremath{\mathfrak{L}}\xspace}
\newcommand{\fkM}{\ensuremath{\mathfrak{M}}\xspace}
\newcommand{\fkN}{\ensuremath{\mathfrak{N}}\xspace}
\newcommand{\fkO}{\ensuremath{\mathfrak{O}}\xspace}
\newcommand{\fkP}{\ensuremath{\mathfrak{P}}\xspace}
\newcommand{\fkQ}{\ensuremath{\mathfrak{Q}}\xspace}
\newcommand{\fkR}{\ensuremath{\mathfrak{R}}\xspace}
\newcommand{\fkS}{\ensuremath{\mathfrak{S}}\xspace}
\newcommand{\fkT}{\ensuremath{\mathfrak{T}}\xspace}
\newcommand{\fkU}{\ensuremath{\mathfrak{U}}\xspace}
\newcommand{\fkV}{\ensuremath{\mathfrak{V}}\xspace}
\newcommand{\fkW}{\ensuremath{\mathfrak{W}}\xspace}
\newcommand{\fkX}{\ensuremath{\mathfrak{X}}\xspace}
\newcommand{\fkY}{\ensuremath{\mathfrak{Y}}\xspace}
\newcommand{\fkZ}{\ensuremath{\mathfrak{Z}}\xspace}




\newcommand{\heart}{{\heartsuit}}
\newcommand{\club}{{\clubsuit}}
\newcommand{\diam}{{\Diamond}}
\newcommand{\spade}{{\spadesuit}}

\newcommand{\bA}{\mathbf A}
\newcommand{\bE}{\mathbf E}
\newcommand{\bG}{\mathbf G}
\newcommand{\bK}{\mathbf K}
\newcommand{\bM}{\mathbf M}
\newcommand{\bQ}{\mathbf Q}



\newcommand{\BA}{\ensuremath{\mathbb {A}}\xspace}
\newcommand{\BB}{\ensuremath{\mathbb {B}}\xspace}
\newcommand{\BC}{\ensuremath{\mathbb {C}}\xspace}
\newcommand{\C}{\BC} % use this a lot
\newcommand{\BD}{\ensuremath{\mathbb {D}}\xspace}
\newcommand{\BE}{\ensuremath{\mathbb {E}}\xspace}
\newcommand{\BF}{\ensuremath{\mathbb {F}}\xspace}
\newcommand{{\BG}}{\ensuremath{\mathbb {G}}\xspace}
\newcommand{\BH}{\ensuremath{\mathbb {H}}\xspace}
\newcommand{\BI}{\ensuremath{\mathbb {I}}\xspace}
\newcommand{\BJ}{\ensuremath{\mathbb {J}}\xspace}
\newcommand{{\BK}}{\ensuremath{\mathbb {K}}\xspace}
\newcommand{\BL}{\ensuremath{\mathbb {L}}\xspace}
\newcommand{\BM}{\ensuremath{\mathbb {M}}\xspace}
\newcommand{\BN}{\ensuremath{\mathbb {N}}\xspace}
\newcommand{\BO}{\ensuremath{\mathbb {O}}\xspace}
\newcommand{\BP}{\ensuremath{\mathbb {P}}\xspace}
\newcommand{\BQ}{\ensuremath{\mathbb {Q}}\xspace}
\newcommand{\BR}{\ensuremath{\mathbb {R}}\xspace}
\newcommand{\BS}{\ensuremath{\mathbb {S}}\xspace}
\newcommand{\BT}{\ensuremath{\mathbb {T}}\xspace}
\newcommand{\BU}{\ensuremath{\mathbb {U}}\xspace}
\newcommand{\BV}{\ensuremath{\mathbb {V}}\xspace}
\newcommand{\BW}{\ensuremath{\mathbb {W}}\xspace}
\newcommand{\BX}{\ensuremath{\mathbb {X}}\xspace}
\newcommand{\BY}{\ensuremath{\mathbb {Y}}\xspace}
\newcommand{\BZ}{\ensuremath{\mathbb {Z}}\xspace}



\newcommand{\CA}{\ensuremath{\mathcal {A}}\xspace}
\newcommand{\CB}{\ensuremath{\mathcal {B}}\xspace}
\newcommand{\CC}{\ensuremath{\mathcal {C}}\xspace}
\newcommand{\CD}{\ensuremath{\mathcal {D}}\xspace}
\newcommand{\CE}{\ensuremath{\mathcal {E}}\xspace}
\newcommand{\CF}{\ensuremath{\mathcal {F}}\xspace}
\newcommand{\CG}{\ensuremath{\mathcal {G}}\xspace}
\newcommand{\CH}{\ensuremath{\mathcal {H}}\xspace}
\newcommand{\CI}{\ensuremath{\mathcal {I}}\xspace}
\newcommand{\CJ}{\ensuremath{\mathcal {J}}\xspace}
\newcommand{\CK}{\ensuremath{\mathcal {K}}\xspace}
\newcommand{\CL}{\ensuremath{\mathcal {L}}\xspace}
\newcommand{\CM}{\ensuremath{\mathcal {M}}\xspace}
\newcommand{\CN}{\ensuremath{\mathcal {N}}\xspace}
\newcommand{\CO}{\ensuremath{\mathcal {O}}\xspace}
\newcommand{\CP}{\ensuremath{\mathcal {P}}\xspace}
\newcommand{\CQ}{\ensuremath{\mathcal {Q}}\xspace}
\newcommand{\CR}{\ensuremath{\mathcal {R}}\xspace}
\newcommand{\CS}{\ensuremath{\mathcal {S}}\xspace}
\newcommand{\CT}{\ensuremath{\mathcal {T}}\xspace}
\newcommand{\CU}{\ensuremath{\mathcal {U}}\xspace}
\newcommand{\CV}{\ensuremath{\mathcal {V}}\xspace}
\newcommand{\CW}{\ensuremath{\mathcal {W}}\xspace}
\newcommand{\CX}{\ensuremath{\mathcal {X}}\xspace}
\newcommand{\CY}{\ensuremath{\mathcal {Y}}\xspace}
\newcommand{\CZ}{\ensuremath{\mathcal {Z}}\xspace}


\newcommand{\RA}{\ensuremath{\mathrm {A}}\xspace}
\newcommand{\RB}{\ensuremath{\mathrm {B}}\xspace}
\newcommand{\RC}{\ensuremath{\mathrm {C}}\xspace}
\newcommand{\RD}{\ensuremath{\mathrm {D}}\xspace}
\newcommand{\RE}{\ensuremath{\mathrm {E}}\xspace}
\newcommand{\RF}{\ensuremath{\mathrm {F}}\xspace}
\newcommand{\RG}{\ensuremath{\mathrm {G}}\xspace}
\newcommand{\RH}{\ensuremath{\mathrm {H}}\xspace}
\newcommand{\RI}{\ensuremath{\mathrm {I}}\xspace}
\newcommand{\RJ}{\ensuremath{\mathrm {J}}\xspace}
\newcommand{\RK}{\ensuremath{\mathrm {K}}\xspace}
\newcommand{\RL}{\ensuremath{\mathrm {L}}\xspace}
\newcommand{\RM}{\ensuremath{\mathrm {M}}\xspace}
\newcommand{\RN}{\ensuremath{\mathrm {N}}\xspace}
\newcommand{\RO}{\ensuremath{\mathrm {O}}\xspace}
\newcommand{\RP}{\ensuremath{\mathrm {P}}\xspace}
\newcommand{\RQ}{\ensuremath{\mathrm {Q}}\xspace}
\newcommand{\RR}{\ensuremath{\mathrm {R}}\xspace}
\newcommand{\RS}{\ensuremath{\mathrm {S}}\xspace}
\newcommand{\RT}{\ensuremath{\mathrm {T}}\xspace}
\newcommand{\RU}{\ensuremath{\mathrm {U}}\xspace}
\newcommand{\RV}{\ensuremath{\mathrm {V}}\xspace}
\newcommand{\RW}{\ensuremath{\mathrm {W}}\xspace}
\newcommand{\RX}{\ensuremath{\mathrm {X}}\xspace}
\newcommand{\RY}{\ensuremath{\mathrm {Y}}\xspace}
\newcommand{\RZ}{\ensuremath{\mathrm {Z}}\xspace}



\newcommand{\ab}{{\mathrm{ab}}}
\newcommand{\Ad}{{\mathrm{Ad}}}
\newcommand{\ad}{{\mathrm{ad}}}
\newcommand{\alb}{{\mathrm{alb}}}
\DeclareMathOperator{\Aut}{Aut}

\newcommand{\Br}{{\mathrm{Br}}}

\newcommand{\cay}{\ensuremath{\operatorname{\fkc_\xi}}\xspace}
\newcommand{\Ch}{{\mathrm{Ch}}}
\DeclareMathOperator{\charac}{char}
\DeclareMathOperator{\Coker}{Coker}
\newcommand{\cod}{{\mathrm{cod}}}
\newcommand{\cont}{{\mathrm{cont}}}
\newcommand{\cl}{{\mathrm{cl}}}
\newcommand{\Cl}{{\mathrm{Cl}}}
\newcommand{\cm}{{\mathrm {cm}}}
\newcommand{\corr}{\mathrm{corr}}

\newcommand{\del}{\operatorname{\partial Orb}}
\DeclareMathOperator{\diag}{diag}
\newcommand{\disc}{{\mathrm{disc}}}
\DeclareMathOperator{\dist}{dist}
\newcommand{\Div}{{\mathrm{Div}}}
\renewcommand{\div}{{\mathrm{div}}}
\newcommand{\DR}{\mathrm{DR}}

\DeclareMathOperator{\End}{End}

\newcommand{\Fil}{\ensuremath{\mathrm{Fil}}\xspace}
\DeclareMathOperator{\Frob}{Frob}

\DeclareMathOperator{\Adm}{Adm}
\DeclareMathOperator{\EO}{EO}
\DeclareMathOperator{\EOfin}{EO_{\rm fin}}






\DeclareMathOperator{\Gal}{Gal}
\newcommand{\Ztwo}{\BZ/2\BZ}
\newcommand{\Zg}{\BZ_{\ge 0}}

\newcommand{\GL}{\mathrm{GL}}
\newcommand{\GLdagger}{\mathrm{GL}^\dagger}
\newcommand{\gl}{\frak{gl}}
\newcommand{\GO}{\mathrm{GO}}
\newcommand{\GSpin}{\mathrm{GSpin}}
\newcommand{\GU}{\mathrm{GU}}
\newcommand{\hg}{{\mathrm{hg}}}
\DeclareMathOperator{\Hom}{Hom}

\newcommand{\id}{\ensuremath{\mathrm{id}}\xspace}
\let\Im\relax
\DeclareMathOperator{\Im}{Im}
\newcommand{\Ind}{{\mathrm{Ind}}}
\newcommand{\inj}{\hookrightarrow}
\newcommand{\Int}{\ensuremath{\mathrm{Int}}\xspace}
\newcommand{\inv}{^{-1}}
\DeclareMathOperator{\Isom}{Isom}

\DeclareMathOperator{\Jac}{Jac}

\DeclareMathOperator{\Ker}{Ker}

\DeclareMathOperator{\Lie}{Lie}
\newcommand{\loc}{\ensuremath{\mathrm{loc}}\xspace}

\newcommand{\M}{\mathrm{M}}
\newcommand{\Mp}{{\mathrm{Mp}}}

\newcommand{\naive}{\ensuremath{\mathrm{naive}}\xspace}
\newcommand{\new}{{\mathrm{new}}}
\DeclareMathOperator{\Nm}{Nm}
\DeclareMathOperator{\NS}{NS}

\newcommand{\OGr}{\mathrm{OGr}}
\DeclareMathOperator{\Orb}{Orb}
\DeclareMathOperator{\ord}{ord}

\DeclareMathOperator{\proj}{proj}

\DeclareMathOperator{\rank}{rank}

\newcommand{\PGL}{{\mathrm{PGL}}}
\DeclareMathOperator{\Pic}{Pic}

\newcommand{\rc}{\ensuremath{\mathrm{rc}}\xspace}
\renewcommand{\Re}{{\mathrm{Re}}}
\newcommand{\red}{\ensuremath{\mathrm{red}}\xspace}
\newcommand{\reg}{{\mathrm{reg}}}
\DeclareMathOperator{\Res}{Res}
\newcommand{\rs}{\ensuremath{\mathrm{rs}}\xspace}

\DeclareMathOperator{\uAut}{\underline{Aut}}
\newcommand{\Sel}{{\mathrm{Sel}}}
%\newcommand{\Sha}{{\underline{\mathrm{|||}}}}
%\newcommand{\Sha}{{\hbox{\cyr Sh}}
\newcommand{\Sim}{{\mathrm{Sim}}}
\newcommand{\SL}{{\mathrm{SL}}}
\DeclareMathOperator{\Spec}{Spec}
\DeclareMathOperator{\Spf}{Spf}
\newcommand{\SO}{{\mathrm{SO}}}
\renewcommand{\O}{{\mathrm{O}}}
\newcommand{\Sp}{{\mathrm{Sp}}}
\newcommand{\SU}{{\mathrm{SU}}}
\DeclareMathOperator{\Sym}{Sym}
\DeclareMathOperator{\sgn}{sgn}

\DeclareMathOperator{\tr}{tr}

\newcommand{\U}{\mathrm{U}}
\newcommand{\ur}{{\mathrm{ur}}}

\DeclareMathOperator{\vol}{vol}



\newcommand{\CCO}{O}



\newcommand{\wt}{\widetilde}
\newcommand{\wh}{\widehat}
\newcommand{\pp}{\frac{\partial\ov\partial}{\pi i}}
\newcommand{\pair}[1]{\langle {#1} \rangle}
\newcommand{\wpair}[1]{\left\{{#1}\right\}}
\newcommand{\intn}[1]{\left( {#1} \right)}
\newcommand{\norm}[1]{\|{#1}\|}
\newcommand{\sfrac}[2]{\left( \frac {#1}{#2}\right)}
\newcommand{\ds}{\displaystyle}
\newcommand{\ov}{\overline}
\newcommand{\incl}{\hookrightarrow}
\newcommand{\lra}{\longrightarrow}
\newcommand{\imp}{\Longrightarrow}
\newcommand{\lto}{\longmapsto}
\newcommand{\bs}{\backslash}


\newcommand{\uF}{\underline{F}}
\newcommand{\ep}{\varepsilon}

%%% some additional macros


\newcommand{\nass}{\noalign{\smallskip}}
\newcommand{\htt}{h}
\newcommand{\cutter}{\medskip\medskip \hrule \medskip\medskip}


\def\tw{\tilde w}
\def\tW{\tilde W}
\def\tS{\tilde \BS}
\def\kk{\mathbf k}
\DeclareMathOperator{\supp}{supp}
% Equation  \AMSname
% Theorem   \theoremname

\def\Gu{[G_u]}  %unipotent classes in G
\def\Gue{[G_u]_e}  %unipotent classes in G, in the image W_e
\def\Gou{[G_{0,u}]}  %unipotent classes in G_0  (can't use 0 nicely in command name, use o instead)
\def\Goue{[G_{0,u}]_e}  %unipotent classes in G_0, in the image of W_e
\def\Wc{[W]}    %conjugacy classes in W
\def\Wec{[W_e]} %elliptic conjugacy classes in W
\def\tPminus{\wt{\CP_{-1}}}
\def\tPplus{\wt{\CP_{1}}}
\def\epsilonmax{\epsilon_{\text{max}}}
\def\leW{\preceq_W}
\def\leu{\preceq_u}
\def\Ftwo{\overline{\BF_2}}
\DeclareMathOperator{\codim}{codim}
% Theorem environments.
%
\newtheorem{theorem}{Theorem}
\newtheorem{proposition}[theorem]{Proposition}
\newtheorem{lemma}[theorem]{Lemma}
\newtheorem {conjecture}[theorem]{Conjecture}
\newtheorem{corollary}[theorem]{Corollary}
\newtheorem{axiom}[theorem]{Axiom}
\theoremstyle{definition}
\newtheorem{definition}[theorem]{Definition}
\newtheorem{example}[theorem]{Example}
\newtheorem{exercise}[theorem]{Exercise}
\newtheorem{situation}[theorem]{Situation}
\newtheorem{remark}[theorem]{Remark}
\newtheorem{remarks}[theorem]{Remarks}
\newtheorem{question}[theorem]{Question}




\numberwithin{equation}{section}
\numberwithin{theorem}{section}






%%%% macros added by Brian
%%%% many of these require the etoolbox package, which should be loaded above

\newcommand{\aform}{\ensuremath{\langle\text{~,~}\rangle}\xspace}
\newcommand{\sform}{\ensuremath{(\text{~,~})}\xspace}

\newcounter{filler}

% gets rid of indentation in itemize and enumerate enivronments, and adds
% a small space between list items:
\setitemize[0]{leftmargin=*,itemsep=\the\smallskipamount}
\setenumerate[0]{leftmargin=*,itemsep=\the\smallskipamount}

% basic right arrow, short in inlines and long in displays
\renewcommand{\to}{%
   \ifbool{@display}{\longrightarrow}{\rightarrow}%
   }
% redefine \mapsto to be short in inlines and long in displays
\let\shortmapsto\mapsto
\renewcommand{\mapsto}{%
   \ifbool{@display}{\longmapsto}{\shortmapsto}%
   }
% stretchable labeled right (2nd is xy-style) & left arrows, well-behaved inline or displayed
\newlength{\olen}
\newlength{\ulen}
\newlength{\xlen}
\newcommand{\xra}[2][]{%
   \ifbool{@display}%
      {\settowidth{\olen}{$\overset{#2}{\longrightarrow}$}%
       \settowidth{\ulen}{$\underset{#1}{\longrightarrow}$}%
       \settowidth{\xlen}{$\xrightarrow[#1]{#2}$}%
       \ifdimgreater{\olen}{\xlen}%
          {\underset{#1}{\overset{#2}{\longrightarrow}}}%
          {\ifdimgreater{\ulen}{\xlen}%
             {\underset{#1}{\overset{#2}{\longrightarrow}}}
             {\xrightarrow[#1]{#2}}}}%
      {\xrightarrow[#1]{#2}}
   }
\makeatother
\newcommand{\xyra}[2][]{%
   \settowidth{\xlen}{$\xrightarrow[#1]{#2}$}%
   \ifbool{@display}%
      {\settowidth{\olen}{$\overset{#2}{\longrightarrow}$}%
       \settowidth{\ulen}{$\underset{#1}{\longrightarrow}$}%
       \ifdimgreater{\olen}{\xlen}%
          {\mathrel{\xymatrix@M=.12ex@C=3.2ex{\ar[r]^-{#2}_-{#1} &}}}%
          {\ifdimgreater{\ulen}{\xlen}%
             {\mathrel{\xymatrix@M=.12ex@C=3.2ex{\ar[r]^-{#2}_-{#1} &}}}
             {\mathrel{\xymatrix@M=.12ex@C=\the\xlen{\ar[r]^-{#2}_-{#1} &}}}}}%
      {\mathrel{\xymatrix@M=.12ex@C=\the\xlen{\ar[r]^-{#2}_-{#1} &}}}%
   }
\makeatletter
\newcommand{\xla}[2][]{%
   \ifbool{@display}%
      {\settowidth{\olen}{$\overset{#2}{\longleftarrow}$}%
       \settowidth{\ulen}{$\underset{#1}{\longleftarrow}$}%
       \settowidth{\xlen}{$\xleftarrow[#1]{#2}$}%
       \ifdimgreater{\olen}{\xlen}%
          {\underset{#1}{\overset{#2}{\longleftarrow}}}%
          {\ifdimgreater{\ulen}{\xlen}%
             {\underset{#1}{\overset{#2}{\longleftarrow}}}
             {\xleftarrow[#1]{#2}}}}%
      {\xleftarrow[#1]{#2}}
   }
% isomorphism arrow, short in inlines and long in displays
\newcommand{\isoarrow}{%
   \ifbool{@display}{\overset{\sim}{\longrightarrow}}{\xrightarrow\sim}%
   }




\begin{document}
\thispagestyle{plain}

\author{Jeffrey Adams}


\date{}                     %% if you don't need date to appear


\title{Map from $[\SO(2n,\C)_u]$ to $[\SO(2n,\Ftwo)_u]$}

%\thanks{}

%\keywords{}
%\subjclass[2010]{}

\date{\today}

\maketitle

\newcommand{\SOC}{[\SO(2n,\C)_u]}
\newcommand{\SOtwo}{[\SO(2n,\Ftwo)_u]}

\section*{Introduction}

In \cite{L1}*{Section 4.1} Lusztig defines an injective map from
$G(\C)$ to $G(\BF_p)$ in terms of the Springer correspondence. He says
that it is dimension preserving. He also says it agrees with the map
defined by Spaltenstein \cite{Spa}*{III.5.2}.
We're interested in the case $G=SO(2n)$.

\bigskip

\subsection{Dimensions of orbits in complex classical groups}

Suppose $\alpha=(\alpha_1,\dots, \alpha_\ell)$ is a partition. Recall
$$
m_\alpha(i)=|\{j\mid \alpha_j=i\}|
$$
is the multipicity function.
Define
$$
\begin{aligned}
  s_\alpha(i)&=\sum_{j\ge i}m_\alpha(j)\\
  &=|\{k\mid \alpha_k\ge i\}|
\end{aligned}
$$
We drop $\alpha$ from the notation when it is clear.

We write $\CC_\alpha$ for the unipotent orbit of a classical group corresponding to $\alpha$.

\begin{lemma}[\cite{collingwood_mcgovern}*{Theorem 6.1.3}]
  \label{l:dim_C}

  \hfil
  \begin{enumerate}
  \item If $G=\Sp(2n,\C)$ then $\dim(\CC_\alpha)=\sum_i s_i^2 +\sum_{\{j\mid r_j\text{ odd}\}}r_j$
    \item If $G=\SO(2n,\C)$ then $\dim(\CC_\alpha)=\sum_i s_i^2 -\sum_{\{j\mid r_j\text{ odd}\}}r_j$
  \end{enumerate}
\end{lemma}


  According to \cite{Spa}*{Section .2.8} the unipotent orbits of
  $\Sp(2n,\Ftwo)$ are parametrized by pairs $(\alpha,\epsilon)$ where
  $\alpha\in\CP_{-1}(2n)$ and $\epsilon$ is a certain function.
  In particular $\epsilon(i)=1$ if $\alpha_i$ is odd, and $\epsilon(i)=0,1$ if $\alpha_i$ is even.

  The unipotent orbits of $\SO(2n,\Ftwo)$ are parametrized by the subset of those for $\Sp(2n,\Ftwo)$ satisfying: $\alpha_1^*$ is even 
($\alpha_1^*$ is even ($\alpha^*$ is the
  transpose partition).

\begin{lemma}[\cite{Spa}*{Section I.2.8}]
  Suppose $(\alpha,\epsilon)$ is the parameter for a unipotent orbit $\CC_{\alpha,\epsilon}$ in $\Sp(2n,\Ftwo)$.
  Let $\CC_\alpha$ be the orbit in $\Sp(2n,\C)$ given by $\alpha$.

  Then 
  $$
  \codim_{\Sp(2n,\Ftwo)}(\CC_{\alpha,\epsilon})=\codim_{\Sp(2n,\C)}(\CC_\alpha)+\sum_{\substack{{i>0\text{ even}}\\{\epsilon(i)=0}}}m_\alpha(i)
  $$

  Suppose $(\alpha,\epsilon)$ is the parameter for a unipotent orbit $\CC_{\alpha,\epsilon}$ in $\SO(2n,\Ftwo)$.
Then
  $$
  \codim_{\SO(2n,\Ftwo)}(\CC_{\alpha,\epsilon})=\codim_{\Sp(2n,\C)}(\CC_\alpha)+\sum_{\substack{{i>0\text{ even}}\\{\epsilon(i)=0}}}m_\alpha(i)-\alpha_1^*
  $$
\end{lemma}

\begin{lemma}
  \label{l:diff_Sp}
  Suppose $(\alpha,\epsilon)$ is the parameter for an orbit $\CC_{\alpha,\epsilon}$ of $\SO(2n,\Ftwo)$.
  Let $\CC_\alpha$ be the orbit for $\Sp(2n,\C)$ given by $\alpha$. Then
  $$
  \dim_{\SO(2n,\Ftwo)}(\CC_{\alpha,\epsilon})=\dim_{\Sp(2n,\C)}(\CC_\alpha)-2n+\alpha_1^*-\sum_{\substack{{i>0\text{ even}}\\{\epsilon(i)=0}}}m_\alpha(i)
  $$
\end{lemma}

This is an elementary reformulation of the previous Lemma.

\begin{lemma}
  \label{l:dim_diff}
  Suppose $\alpha\in\CP(2n)$, and all rows have even multiplicity, so $\alpha\in \CP_{1}(2n)\cap \CP_{-1}(2n)$.
  Suppose $(\alpha,\epsilon)$ is the parameter for an orbit $\CC_{\alpha,\epsilon}$ of $\SO(2n,\Ftwo)$,
  and let $\CC_\alpha$ be the orbit for $\SO(2n,\C)$ corresponding to $\alpha$. Then
  $$
  \dim_{\SO(2n,\Ftwo)}(\CC_{\alpha,\epsilon})=\dim_{\SO(2n,\C)}(\CC_\alpha) +\alpha_1^* -\sum_{i\text{ odd}}m_\alpha(i)
 -\sum_{\substack{i>0\text{ even}\\\epsilon(i)=0}}m_\alpha(i)
  $$

  In particular
  $$
   \dim_{\SO(2n,\Ftwo)}(\CC_{\alpha,\epsilon})=\dim_{\SO(2n,\C)}(\CC_\alpha)\Leftrightarrow \epsilon(i)=0\text{ for all $i$ even}
  $$
\end{lemma}

\begin{proof}
  The first part is elementary from the preceding Lemma and Lemma \ref{l:dim_C}.
  For the second note that $\sum_i m_\alpha(i)=\alpha_1^*$.
\end{proof}
\begin{lemma}
\label{l:cone}
If $G=\SO(2n,\C)$  or $\SO(2n,\Ftwo)$ the unipotent cone has dimension $2n^2-2n$.
\end{lemma}

Maybe this is obvious from the algebraic groups point of view? In any event here's a proof.

\begin{proof}
  This is well known for $\SO(2n,\C)$.

  The principal unipotent orbit $\CC_{princ}$ for $\SO(2n,\Ftwo)$ is $([2n-2,2],*)$.
  By Lemma \ref{l:dim_C} the orbit $([2n-2,2])$ for $\Sp(2n,\C)$  has dimension $2n-2$.
  By the Lemma \ref{l:diff_Sp} this gives
  $$
  \dim_{\SO(2n,\Ftwo)}(\CC_{princ})=(2n^2-2)+2-2n-0=2n^2-2n
  $$
\end{proof}



\subsection{Spaltenstein's map}

We'll only discuss the case of $\SO(2n)$.

Suppose $\alpha\in \CP_{1}(2n)$ so $\CC_\alpha\in \SOC$.
Then $\pi_S(\CC_\alpha)=\CC_{\alpha',\epsilon}$
as defined in \cite{Spa}*{III.7}.
Here $\alpha'\in\CP_{-1}(2n)$ is the collapse of $\alpha$ onto $\CP_{-1}(2n)$,
and
$\epsilon(i)=0$ if and only if $i$ is even, $\alpha_i^*,\alpha_{i+1}^*$ are even, and $\alpha_i^*\ne \alpha_{i+1}^*$.

Let $\CU(X)$ denote the unipotent elements of $X$. 

\begin{lemma}[\cite{Spa}*{Theorem 5.2}]
Suppose $\pi_S(\CC)=\CC'$. Then  
$$
\codim_{\CU(\SO(2n,\CC))}(\CC)=
\codim_{\CU(\SO(2n,\Ftwo))}(\CC')=
$$
\end{lemma}

Note that the codimension is in the unipotent cone. By Lemma \ref{l:cone} these cones have the same dimension.
Therefore

\begin{lemma}
  Let $\CC$ be a unipotent orbit for $\SO(2n,\C)$. Then
$$
\dim(\pi_S(\CC))=\dim(\CC)
$$
\end{lemma}

This is clear if there is no collapsing.

\begin{example}
  \label{e:even}
  Suppose $\alpha\in\CP_{-1}(2n)\cap \CP_1(2n)$, i.e. all parts have even multiplicity.
  Let $\CC=\CC_{\alpha}$ for $\SO(2n,\C)$. Then $\pi_S(\CC)=\CC_{\alpha,\epsilon}$,
  with the same $\alpha$ and some $\epsilon$. 

  By Lemma the second part of \ref{l:dim_diff} the assertion holds if
  and only if $\epsilon(i)=0$ for all $i$ even. This holds by the description
  of $\epsilon$ in \cite{Spa}*{Section III.7.2}.
\end{example}

\subsection{An example}

Consider the group $SO(12)$, and the elliptic class given by the partition $[2,2,1,1]$ of $6$.
According to Lusztig this maps to $\CC_\alpha$ for $\SO(12,\C)$ where $\alpha=[5,3,3,1]$.
This is using Lusztig's $\psi$: if $\alpha=[2,2,1,1]$ then $\psi[1,-1,1,-1]$.
This orbit has dimension $50$.

According to Lusztig this class goes to
$([4,4,2,2],\epsilon(4)=\epsilon(2)=1)$ for $\SO(12,\Ftwo)$, and this  orbit has dimension $50$.
Note that if we change $\epsilon$ the
dimension goes down: we can  get dimension $50,48,48,46$ for various choices of $\epsilon$.

According to Spaltenstein $[5,3,3,1]$ goes to
$([4,4,2,2],\epsilon(4)=\epsilon(2)=1)$ for $\SO(12,\Ftwo)$, exactly
the same as in Lusztig's assertion. A key point is that while
Spaltenstein's map sometimes produces $\epsilon$ with $\epsilon(i)=0$,
that is not the case here. And apparently this never happens for orbits in $\SO(2n,\C)$
coming from $\Wec$.

Side note: Spaltenstein's map takes $[4,4,2,2]$ to
$([4,4,2,2],\epsilon(4)=0,\epsilon(2)=0)$, and this has dimension
$46$, again as stated by Spaltenstein, Example \ref{e:even}, and
calculated directly. But this doesn't arise from $\Wec$.

\subsection{Relation between the Lusztig and Spaltenstein maps}

I think everything comes down to the following Conjecture, which is purely combinatorial.

\begin{conjecture}
Suppose $\beta$ is a partition of $n$ with an even number of parts.
Set $\alpha=2\beta+\psi(\alpha)\in \CP(2n)$.
Then
\begin{enumerate}
\item $\alpha\in\CP_{1}(2n)$
\item The $\CP_{-1}$-collapse of $\alpha$ is $2\beta$.
\item $\pi_S(\CC_\alpha)=(2\beta,\epsilon)$ where $\epsilon(i)=1$ for all $i$.
\end{enumerate}
\end{conjecture}

Assuming this is true then the Lusztig map does indeed agree with the Spaltenstein
map, at least on the image of $\Wec$.

\begin{thebibliography}{AAAA99}

\bibitem[BB05]{bb}
  F. Brenti and A. Bj\"{o}rner, \emph{Combinatorics of Coxeter groups},
  Graduate Texts in Mathematics, vol. 231, Springer, New York (2005)

\bibitem[CM93]{collingwood_mcgovern}
  D. Collingwood and M. McGovern, \emph{Nilpotent orbits in semisimple Lie algebras},
  Van Nostrand Reinhold Mathematics Series, Van Nostrand Reinhold Co., New York (1993)

\bibitem[DHM13]{DHM}
O. Dudas, X. He and J. Michel, \emph{Private communication}, 2013.

\bibitem[DL76]{DL}
P. Deligne and G. Lusztig, \emph{Representations of reductive groups over finite fields}, Ann. of Math. (2) 103 (1976), no. 1, 103--161.

\bibitem[DM15]{DM}
O. Dudas and G. Malle, \emph{Decomposition matrices for low-rank unitary groups}, Proc. Lond. Math. Soc. (3) 110 (2015), no. 6, 1517--1557.

\bibitem[JO00]{JO}
A. J. de Jong and F. Oort, \emph{Purity of the stratification by Newton polygons}, J. Amer. Math. Soc. 13 (2000), 209--241.

\bibitem[Lu11]{L1}
G. Lusztig, \emph{From conjugacy classes in the Weyl group to unipotent classes}, Represent. Theory 15 (2011), 494--530.

\bibitem[Lu12]{L3}
G. Lusztig, \emph{From conjugacy classes in the Weyl group to unipotent classes, III}, Represent. Theory 16 (2012), 450--488.

\bibitem[Ha15]{Ha}
P. Hamacher, \emph{The geometry of Newton strata in the reduction modulo p of Shimura varieties of PEL type}, Duke Math. J. 164 (2015), no. 15, 2809--2895.

\bibitem[HV11]{HV}
U. Hartl and E. Viehmann, \emph{The Newton stratification on deformations of local G-shtukas}, J. Reine Angew. Math. 656 (2011), 87--129.

\bibitem[He07]{He07}
X. He, {\em Minimal length elements in some double cosets of Coxeter groups},  Adv. Math. 215 (2007), 469--503.

\bibitem[He14]{He-Ann}
X. He, \emph{Geometric and homological properties of affine Deligne-Lusztig varieties}, Ann. of Math. (2) 179 (2014), 367--404.

\bibitem[He16a]{He-CDM}
X. He, \emph{Hecke algebras and $p$-adic groups}, Current developments in mathematics 2015, 73--135, Int. Press, Somerville, MA, 2016.

\bibitem[He16b]{He-KR}
X. He, \emph{Kottwitz-Rapoport conjecture on unions of affine Deligne-Lusztig varieties},  Ann. Sci. \`Ecole Norm. Sup. 49 (2016), 1125--1141.

\bibitem[HN14]{HN2}
X. He and S. Nie, \emph{Minimal length elements of extended affine Weyl group}, Compos. Math. 150 (2014), 1903--1927.

\bibitem[Sp82]{Spa}
N. Spaltenstein, \emph{Classes unipotents et sous-groupes de Borel}, Lecture Notes in Math. 946, Springer, 1982.

\bibitem[Vi13]{Vi}
E. Viehmann, \emph{Newton strata in the loop group of a reductive group}, Amer. J. Math. 135 (2013), no. 2, 499--518.

\bibitem[Wa63]{Wall}
G. E. Wall, \emph{On the conjugacy classes in the unitary, symplectic and orthogonal groups},  J. Austral. Math. Soc., 3 (1963), 1--63.

\end{thebibliography}


\end{document}
